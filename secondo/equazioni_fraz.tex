\chapter{Equazioni frazionarie di primo grado}
\label{cha:Equazionefrazionariaprimogrado}
\section{Definizioni}
\label{sec:definizioni}
Una equazione è frazionaria\index{Equazione!frazionaria} se l'incognita compare al denominatore.
\begin{esempio}
L'equazione seguente non è frazionaria
\[3(x+1)+(x-2)(x-\dfrac{1}{2}=0) \]
mentre
\[\dfrac{3x+2}{4x+2}+x=0 \]
è un'equazione frazionaria l'incognita è al denominatore.
\end{esempio}
Una frazione è una divisione e in una divisione non è possibile dividere per zero. Può accadere  che vi siano valori dell'incognita che rendono il denominatore uguale a zero e di conseguente impossibile la frazione. Il dominio di una funzione o campo di esistenza è l'insieme dei valori per cui la frazione esiste.
\begin{enumerate}
\item Per ogni frazione che contengono le incognite discuto i denominatori.
	\begin{itemize}
	\item Pongo i denominatori uguali a zero e risolvo l'equazione che ottengo.
	\item Escludo i valori trovati negandoli $\neq$
	\end{itemize}
	\item  Scompongo il fattori primi i denominatori (attenzione alla differenza fra fattori ed addendi) es: $2x$ e  $2x+1$ sono due fattori fra loro diversi.
	\item Calcolo il mcm (Fattori comuni e non comuni, presi una sola volta con il massimo esponente)
	\item Traccio la linea di frazione 
	\item Per ogni frazione divido il minimo comune multiplo per il denominatore e il risultato della divisione lo moltiplico per il numeratore ricordando che sono obbligatorie le parentesi quando 
	\begin{itemize}
	\item Moltiplico fra loro polinomi
	\item Davanti alla linea di frazione vi è un segno meno
	\end{itemize}
	\item Ottengo un unica frazione che semplifico togliendo il denominatore
	\item Eseguo i calcoli e separo le incognite che scrivo sinistra, dai numeri che scrivo a destra, ricordando che  se un termine viene spostato rispetto all'uguale cambia di segno. Attenzione Se un numero moltiplica una lettere es $2x$ è un'incognita e andrà a sinistra ,diverso da $2$ che andra a destra.
	\item Sommo fra di loro le incognite e fra di loro i numeri.
	\item Ottengo 
	\begin{itemize}
	\item Un'equazione di primo grado che risolvo dividendo  a sinistra e a destra per il numero davanti all'incognita. Attenzione ogni numero ha un segno che non può essere trascurato.
	\begin{itemize}
	\item Controllo se i risultati ottenuti  sono accettabili confrontandoli con i valori che ho escluso eventualmente scartandoli nel caso fossero uguali.
	\end{itemize}
	\item Un'uguaglianza impossibile. Esempio $0=2$
	\item Un'identità. Esempio $2=2$
	\end{itemize}
.\end{enumerate}
