\chapter{Equazioni}
\label{sec:equazioni}
\section{Definizioni}
\begin{definizionet}{Definizione di equazione}{}
Un'equazione\index{Equazione} è l'uguaglianza fra due espressioni.
\end{definizionet}
 Dato che dipende dai valori che vengono assegnati alle lettere un'equazione è un'uguaglianza condizionata.
\begin{definizionet}{Soluzioni}{}
  I valori, che sostituiti alle lettere rendono vere l'uguaglianza,  sono chiamate soluzioni\index{Equazione!soluzione}
\end{definizionet}
\begin{esempiot}{Uguaglianze}{}
Classificare le seguenti uguaglianze
\end{esempiot}
\begin{itemize}
\item $2+3=3+5$ non è un'equazione. Mancano le lettere.
\item $2+x=3x+5y$ è un'equazione. 
\item $z+3x=0$ è un'equazione.
\end{itemize}
\begin{esempiot}{Soluzioni}{}
Verificare se $x=7$ è soluzione di
 l'equazione $2x-5=x+2$ 
\end{esempiot} 
 $x=7$ è soluzione infatti $2\cdot 7-5=7+2$ $9=9$ l'uguaglianza è verificata. Mentre per $x=3$ $2\cdot 3-5=7+2$ $1\neq5$ quindi non è soluzione.

Le lettere sono chiamate variabili\index{Equazione!variabile} o costanti\index{Equazione!costante}. Una variabile o incognita è una quantità non nota a priori che può assumere qualunque valore. Una costante è una quantità non nota ma fissa. Di solito si usano $x,y,z$ per indicare le incognite $a,b,c,\dots$ per indicare le costanti.  

Un'equazione in cui compare una sola lettera è detta in una incognita, con due diverse in due incognite eccetera. Il segno di uguaglianza divide l'equazione in due parti, la parte a sinistra chiamata primo membro, la parte a destra secondo membro.
\section{Principi di equivalenza}
\begin{definizionet}{Principio di equivalenza}{}
Due o più equazioni sono equivalenti\index{Equazione!equivalente} se hanno le stesse soluzioni.
\end{definizionet}
\begin{esempiot}{Equivalenza equazioni}{}
Verificare se le due equazioni sono equivalenti 
\begin{align*}
5x+4&=4x+3\\
3x+2&=2x+3
\end{align*}
\end{esempiot}
le equazioni 
hanno la stessa soluzione
\begin{align*}
5x+4&=4x+3\\
5\cdot(-1)+4&=4\cdot(-1)+3\\
-1&=-1\\
3x+2&=2x+3\\
3\cdot(-1)+2&=2\cdot(-1)+3\\
-1&=-1
\end{align*}
Quindi le due equazioni sono equivalenti.

\subsection{Primo principio di equivalenza}
\label{sec:PrimoprincipioEquivalenza}
\begin{principiot}{Primo principio}{}
Se aggiungiamo o togliamo la stessa quantità\footnote{Quantità definita} al primo e al secondo membro di una equazione,  otteniamo un'equazione  equivalente\index{Equazione!equivalente} a quella di partenza.
\end{principiot}
\begin{esempiot}{Primo principio equivalenza}{}
$8x+14=6x+10$
\end{esempiot}
\begin{align*}
8x+14&=6x+10
\intertext{aggiungendo $+10$ ad entrambi i membri}
8x+14+10&=6x+10+10\\
8x+24&=6x+20
\intertext{le due equazioni sono equivalenti infatti}
8\cdot(-2)+14&=6\cdot(-2)+10\\
8&=8\\
8\cdot(-2)+14&=6\cdot(-2)+10\\
-2&=-2
\intertext{quindi $-2$ è soluzione per entrambe}
\end{align*}
\subsection{Conseguenze primo principio}
Se un termine passa  dal primo al secondo membro di una equazione o viceversa cambia di segno\index{Regola!del trasporto}.
\begin{esempiot}{Conseguenze primo principio}{}
$ x+5 = 8$
\end{esempiot}
\begin{NodesList}[dy=5pt,margin=3cm]
 \[ % formula no "inline"
 \begin{aligned}
 x+5 &= 8 \AddNode\\
 x +5-5 &= 8-5 \AddNode\\
 x + 0 &= 8-5 \AddNode
 \end{aligned}
 \]
 \LinkNodes{\begin{minipage}{2cm}
aggiungo $-5$ ad entrambi i membri
 \end{minipage}
 }
 \LinkNodes{ $5-5=0$ }
 \end{NodesList}
 in pratica
 \begin{NodesList}[dy=5pt,margin=3cm]
  \[ % formula no "inline"
  \begin{aligned}
  x+5 &= 8 \AddNode\\
  x  &= 8-5 \AddNode
  \end{aligned}
  \]
  \LinkNodes{\begin{minipage}{2cm}
 sposto e cambio di segno
  \end{minipage}
  }
  \end{NodesList}
Se la stessa quantità è presente nel primo o secondo membro dell'equazione allora può essere eliminata\index{Regola!cancellazione}.
\begin{esempiot}{Regola di cancellazione}{}
$ x+5 = 8+5$
\end{esempiot}
\begin{NodesList}[dy=5pt,margin=3cm]
 \[ % formula no "inline"
 \begin{aligned}
 x+5 &= 8+5 \AddNode\\
 x +5-5 &= 8+5-5 \AddNode\\
 x + 0 &= 8+0 \AddNode
 \end{aligned}
 \]
 \LinkNodes{\begin{minipage}{2cm}
aggiungo $-5$ ad entrambi i membri
 \end{minipage}
 }
 \LinkNodes{ $5-5=0$ }
 \end{NodesList}
 in pratica
 \begin{NodesList}[dy=5pt,margin=3cm]
  \[ % formula no "inline"
  \begin{aligned}
  x+5 &= 8+5 \AddNode\\
  x  &= 8 \AddNode
  \end{aligned}
  \]
  \LinkNodes{\begin{minipage}{2cm}
semplifico
  \end{minipage}
  }
  \end{NodesList}

\subsection{Secondo principio di equivalenza}
\label{sec:SecondoprincipioEquivalenza}
\begin{principiot}{Secondo principio}{}
Se moltiplichiamo o dividiamo per  la stessa quantità diversa da zero il primo e il secondo membro di una equazione,  otteniamo un'equazione  equivalente\index{Equazione!equivalente} a quella di partenza.
\end{principiot}
\begin{esempiot}{Secondo principio di equivalenza}{}
$2x+2=x+5$
\end{esempiot}
\begin{align*}
2x+2&=x+5
\intertext{moltiplico per  $+5$  entrambi i membri}
5\cdot(2x+2)&=5\cdot(x+5)\\
10x+10&=5x+25
\intertext{le due sono equivalenti infatti}
2\cdot(3)+2&=3+5\\
8&=8\\
10\cdot(3)+10&=5\cdot(3)+25\\
40&=40
\intertext{quindi $-2$ è soluzione per entrambe}
\end{align*}
\section{Forma normale}
\label{sec:formanormale}
\begin{definizionet} {}{}
Dopo aver trasportato a primo membro tutti i termini di una equazione si ottiene un polinomio ordinato e l'equazione diventa \[P(x)=0\]
In questo caso l'equazione\index{Equazione!forma normale} si dice in forma normale.
\end{definizionet}
\begin{esempiot}{Forma normale}{}
L'equazione\[3x^2+3=0\] è in forma normale.

L'equazione\[(x-2)(3x+2)+5x=0\] non è in forma normale.
\end{esempiot}
Il grado più grande dell'equazione rispetto all'incognita è detto grado dell'equazione\index{Equazione!grado}.
\section{Equazioni di primo grado}
\label{sec:equazionidiprimogrado}
\begin{definizionet}{}{}
Una equazione\index{Equazione!di primo grado} di primo grado è un'equazione della forma \[ax=b\]
$a,b\in\R$
\end{definizionet}
\begin{esempiot}{Equazione di promo grado}{}
Risolvere l'equazione $8x+2(x+11) = 6x+3(x-3) $
\end{esempiot}
 \begin{NodesList}[margin=3cm]
  \begin{align*}
  8x+2(x+11) = 6x+3(x-3) \AddNode\\
  8x+2x+2  = 6x+3x-9 \AddNode\\
  \intertext{\hfil isolati i termini con l'incognita \hfil}
  8x+2x-6x-3x  = -2-9 \AddNode\\
    x  = -11 \AddNode
  \end{align*}
  \LinkNodes{Moltiplico}
  \LinkNodes{ }
  \LinkNodes{ Ottengo la soluzione}
  \end{NodesList}
  \[x=-11\]
  è soluzione
Nell'esempio che segue l'equazione è scritta in maniera più complessa. Questo impone delle priorità nella risoluzione della stessa. 
\begin{esempiot}{Equazione di primo grado}{}
Risolvere l'equazione \[2(x+\dfrac{4}{3})-\dfrac{5x-3}{2}=2x+3(x+2) \]
\end{esempiot}
 \begin{NodesList}[margin=3cm]
  \begin{align*}
  2\overbrace{(x+\dfrac{4}{3})}-\dfrac{5x-3}{2}=2x+\overbrace{3(x+2)} \AddNode\\
%  \intertext{\hfil isolati i termini con l'incognita \hfil}
\overbrace{2(\dfrac{3x+4}{3})}-\dfrac{5x-3}{2}=\overbrace{2x+3x}+6 \AddNode\\
  \dfrac{6x+8}{3}-\dfrac{5x-3}{2}=5x+6   \AddNode\\
 \dfrac{12x+16-15+9=30x+31}{6}   \AddNode\\
 -3x+25=30x+31\AddNode\\
  \intertext{\hfil isolati i termini con l'incognita \hfil}
 \overbrace{-3x-30x}=\overbrace{-25+31}\AddNode\\
 -33x=6\AddNode\\
 x=-\dfrac{6}{33}\AddNode\\
 x=-\dfrac{2}{11}\AddNode\\
  \end{align*}
  \LinkNodes{Precedenze}
   \LinkNodes{Moltiplico e sommo}
  \LinkNodes{mcm}
  \LinkNodes{Moltiplico per $6$}
  \LinkNodes{Sommo}
  \LinkNodes{Separo}
  \LinkNodes{Divido}
    \LinkNodes{Semplifico}
  \end{NodesList}
Le due equazioni precedenti, in origine, non erano in forma normale\index{Equazione!forma normale}, semplificando e separando le variabili otteniamo un'equazione in forma normale,  che viene risolta dividendo per il termine davanti l'incognita. Dato che abbiamo ottenuto una soluzione l'equazione è determinata\index{Equazione!determinata}.
 
\begin{esempiot}{Equazione di primo grado}{}
Risolvere l'equazione \[2(3x+2)=3(\dfrac{4}{3}x-1)+2(x+1) \]
\end{esempiot}
 \begin{NodesList}[margin=3cm]
  \begin{align*}
\overbrace{2(3x+2)}=3(\dfrac{4}{3}x-1)+\overbrace{2(x+1)} \AddNode\\
%  \intertext{\hfil isolati i termini con l'incognita \hfil}
6x+3=\overbrace{3(\dfrac{4x-3}{3})}+2x+2 \AddNode\\
 6x+3=4x-3+2x+2  \AddNode\\
  \intertext{\hfil isolati i termini con l'incognita \hfil}
 \overbrace{6x-4x-2x}=\overbrace{-4-3+2}  \AddNode\\
 0=-5\AddNode
  \end{align*}
  \LinkNodes{Precedenze}
   \LinkNodes{Moltiplico}
  \LinkNodes{Separo}
  \LinkNodes{Sommo}
  \end{NodesList}
  a  primo membro abbiamo zero al secondo meno cinque. L'uguaglianza è impossibile l'equazione è impossibile.
\begin{figure}
	\centering
	\includestandalone[width=.3\linewidth]{secondo/diagrammi/AlberoBinario1}
	\caption[]{Classificazione equazioni}
	\label{fig:AlberoBinarioeqa1}
\end{figure}

Anche ne caso che segue l'incognita scompare solo che cambia il tipo della soluzione. Nell'esempio che segue scompare l'incognita ma l'uguaglianza ottenuta non è sempre falsa ma sempre vera. L'uguaglianza è un'identità.
\begin{esempiot}{}{}
Risolviamo l'equazione \[ 6(x-3) = 3(x-1)+5(x+\dfrac{2}{5})-(2x+17)\]
\end{esempiot}
\begin{NodesList}[margin=3cm]
  \begin{align*}
\overbrace{6(x-3)} = \overbrace{3(x-1)}+5(\overbrace{x+\dfrac{2}{5}})\overbrace{-(2x+17)} \AddNode\\
%  \intertext{\hfil isolati i termini con l'incognita \hfil}
6x-18 = 3x-3+\overbrace{5(\dfrac{5x+2}{5})}-2x-17\AddNode\\
6x-18 = 3x-3+5x+2-2x-17  \AddNode\\
  \intertext{\hfil isolati i termini con l'incognita \hfil}
 \overbrace{6x-3x-5x+2x}=\overbrace{18-3+2-17}  \AddNode\\
 0=0\AddNode
  \end{align*}
  \LinkNodes{Precedenze}
   \LinkNodes{Moltiplico}
  \LinkNodes{Separo}
  \LinkNodes{Sommo}
  \end{NodesList}
  Il primo membro è uguale al secondo l'uguaglianza è sempre vera. 







%\begin{table}[H]
%\centering
%\begin{tabular}{LCR}
%\toprule
%+a&=&\ldots\\
%\ldots&=&-a\\
%\bottomrule
%\end{tabular}
%\caption{Regola del trasporto}
%\label{tab:regtrasporto}
%\end{table}
%\begin{table}[H]
%\centering
%\begin{tabular}{LCR}
%\toprule
%\dfrac{\cdots\cdots}{a}&=&\dfrac{\cdots\cdots}{a}\\
%&\\
%a\cdot\dfrac{\cdots\cdots}{a}&=&a\cdot\dfrac{\cdots\cdots}{a}\\
%&\\
%\cdots\cdots&=&\cdots\cdots\\
%\bottomrule
%\end{tabular}
%\caption{Semplificazione denominatore}
%\label{tab:Semplificazionedenominatore}
%\end{table}
%\begin{table}[H]
%
%\centering
%\begin{tabular}{CCCCL}
%\toprule
%\multicolumn{5}{c}{ax=b}\\
%\hline
%%&\\
%\multicolumn{2}{c}{coefficienti}&&soluzione&tipo soluzione\\
%\midrule
%a\neq0&b\neq0&ax=b&x=\dfrac{b}{a}&determinata\\
%%&\\
%a\neq0&b=0&ax=0&x=0&determinata\\
%%&\\
%a=0&b=0&0x=0&&indeterminata\\
%%&\\
%a=0&b\neq0&0x=b&&impossibile\\
%\bottomrule	
%\end{tabular}
%\caption{Soluzioni equazioni primo grado intere}
%\label{tab:equazioniprimogrado}
%\end{table}
%\begin{table}%
%
%\centering
%\begin{tabular}{LR}
%\toprule
%Tipo&Nome\\
%\midrule
%ax^2+c=0&Pura\\
%\hline
%\multicolumn{2}{c}{Risoluzione}\\
%\multicolumn{2}{C}{ax^2=-c}\\
%\multicolumn{2}{C}{x^2=-\dfrac{c}{a}}\\
%\multirow{3}*{Se $-\dfrac{c}{a}>0$ esistono soluzioni reali} &x_1=-\sqrt{-\dfrac{c}{a}}\\
%&\\
%&x_2=+\sqrt{-\dfrac{c}{a}}\\
%&\\
%Se -\dfrac{c}{a}<0\text{ non esistono soluzioni reali}&\\
%&\\
%\bottomrule	
%%\end{tabular}
%%\caption{Equazione secondo grado pura}
%%\label{tab:equazione2GradoPura}
%%\end{table}
%%\begin{table}%
%%
%%\centering
%%\begin{tabular}{LR}
%\toprule
%Tipo&Nome\\
%\midrule
%ax^2+bx=0&Spuria\\
%\hline
%\multicolumn{2}{c}{Risoluzione}\\
%\multicolumn{2}{C}{ax^2+bx=0}\\
%\multicolumn{2}{C}{x(ax+b)=0}\\
%\multicolumn{2}{C}{x_1=0}\\
%\multicolumn{2}{C}{ax+b=0}\\
%\multicolumn{2}{C}{x_2=-\dfrac{b}{a}}\\
%\bottomrule	
%%\end{tabular}
%%\caption{Equazione secondo grado spuria}
%%\label{tab:equazione2GradoSpuria}
%%\end{table}
%%\begin{table}%
%%
%%\centering
%%\begin{tabular}{LR}
%\toprule
%Tipo&Nome\\
%\midrule
%ax^2=0&Monomia\\
%\hline
%\multicolumn{2}{c}{Risoluzione}\\
%\multicolumn{2}{C}{ax^2=0}\\
%\multicolumn{2}{C}{x_1=0}\\
%\multicolumn{2}{C}{x_2=0}\\
%\bottomrule	
%%\end{tabular}
%%\caption{Equazione secondo grado monomia}
%%\label{tab:equazione2GradoMonomia}
%%\end{table}
%%\begin{table}%
%%
%%\centering
%%\begin{tabular}{LR}
%\toprule
%Tipo&Nome\\%
%\midrule
%ax^2+bx+c=0&Completa\\%
%\hline
%\multicolumn{2}{c}{Risoluzione}\\%
%\multirow{3}*{$b^2-4ac>0$}&x_1=\dfrac{-b+\sqrt{b^2-4ac}}{2a}\\%
%&\\
%&x_2=\dfrac{-b-\sqrt{b^2-4ac}}{2a}\\%
%\hline
%\multirow{3}*{$b^2-4ac=0$}&x_1=-\dfrac{b}{2a}\\%
%&\\
%&x_2=-\dfrac{b}{2a}\\%
%\hline
%\multirow{3}*{$b^2-4ac<0$}&\\
%&\text{nessuna soluzione reale}\\%
%&\\
%\bottomrule	
%\end{tabular}
%\caption{Equazioni secondo grado}
%\label{tab:equazione2Gradoelenco}
%\end{table}
