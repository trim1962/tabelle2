\chapter{Sistemi lineari in due incognite}
\label{sec:sistemiLineariInDueIncognite}
\section{Equazioni in più variabili}
\label{sec:EquazioniInPiuVariabili}
\begin{definizionet}{Equazione in più variabili}{}\index{Equazione!più variabili!definizione}
	Un'equazione in più variabili è un'equazione in più incognite.
\end{definizionet}
\begin{definizionet}{Definizione di soluzione}{}\index{Equazione!più variabili!soluzione}
Per un'equazione in più variabili, una soluzione, è un insieme ordinato di valori che verificano l'equazione.
\end{definizionet}
\begin{esempiot}{}{}
	Risolviamo l'equazione\[ 3x+y=7\]
\end{esempiot}	
Questa equazione, in due incognite, ha per soluzione
$x=2$ e $y=1$ infatti \[3\cdot2+1=7\]
Ma anche $x=3$ e $y=-2$ infatti \[3\cdot3-2=7\] è soluzione.
In genere un'equazione in più incognite ha più di una soluzione.
\begin{figure}
	\centering
\includestandalone[width=.9\textwidth]{secondo/sistemi/mappe_concettuali_sistema_1}
	\caption{Classificazione di un sistema}
	\label{fig:ClassificazioneDiUnSistema}
\end{figure}
\section{Sistemi}
\label{sec:Sistemi}
Più  equazioni  possono avere le stesse soluzioni. Un sistema è un'insieme di due o più equazioni. Risolvere un sistema è verificare che più equazioni hanno le stesse soluzioni.
\begin{definizionet}{Definizione di sistema}{}\index{Sistema!definizione}
Un sistema è un'insieme di due o più  equazioni.
\end{definizionet}
\begin{definizionet}{Definizione di grado di sistema}{}\index{Sistema!grado}\label{def:sistemaGrado}
Il grado di un sistema, di più equazioni in tante incognite,  è il prodotto dei gradi delle equazioni del sistema ridotte in forma normale.
\end{definizionet}
\begin{definizionet}{Definizione di soluzione}{}\index{Sistema!definizione!soluzione}
Una soluzione per un sistema è un insieme ordinato di valori che sono soluzione per ogni equazione del sistema.
\end{definizionet}
\section[Classificazione rispetto alle soluzioni]{Classificazione dei sistemi rispetto alle soluzioni}
\begin{definizionet}{Definizione di equivalenza}{}\index{Sistema!definizione di equivalenza}
Due sistemi sono equivalenti quando hanno lo stesso insieme soluzione.
\end{definizionet}
\begin{definizionet}{Forma normale di un sistema lineare}{}\index{Sistema!lineare!forma normale}{}
Un sistema di due equazioni di 1 grado in due incognite x, y, a coefficienti numerici, si dice ridotto in forma normale se è del tipo
\[\left\{\begin{array}{l} {ax+by=c} \\ {a'x+b'y=c'}\end{array}\right. \]
dove a, b, a', b' si chiamano coefficienti delle incognite c, c' si chiamano termini noti
\end{definizionet}
\begin{teoremat}{Teorema fondamentale}{}\index{Sistema!lineare!teorema fondamentale}{}
Se i coefficienti del sistema lineare sono diversi da zero e non sono tra loro proporzionali $\dfrac{a}{a'} \ne \dfrac{b}{b'}$ il sistema ammette una e una sola soluzione data da
\[
\begin{cases}
	x=\dfrac{b'c-bc'}{ab'-a'b}\\
	y=\dfrac{ac'-a'c}{ab'-a'b}
\end{cases}
\]
Se, invece, sono proporzionali solo i coefficienti delle incognite cioè; $\dfrac{a'}{a} =\dfrac{b'}{b} \ne \dfrac{c'}{c} $, allora il sistema è impossibile
Se, invece, sono proporzionali i coefficienti e i termini noti delle equazioni cioè; $\dfrac{a'}{a} =\dfrac{b'}{b} =\dfrac{c'}{c} $, allora il sistema è indeterminato.
\end{teoremat}
Consideriamo il sistema canonico, \[ \left\{\begin{array}{l} {ax+by=c} \\ {a'x+b'y=c'} \end{array}\right. \]
moltiplicando la prima equazione per b' e la seconda per --b, avremo  \[\left\{\begin{array}{l} {ab'x+bb'y=b'c} \\ {-ba'x-bb'y=-bc'} \end{array}\right. \]
sommando lungo le colonne otteniamo \[\dfrac{\left\{\begin{array}{l} {ab'x+bb'y=b'c} \\ {-ba'x-bb'y=-bc'} \end{array}\right. }{\left(ab'-ba'\right)x=b'c-bc'} \]  da cui  \[x=\dfrac{b'c-bc'}{ab'-a'b} \]
moltiplicando la prima equazione per a' e la seconda per --a avremo  \[\left\{\begin{array}{l} {aa'x+ba'y=a'c} \\ {-aa'x-ab'y=-ac'} \end{array}\right. \] sommando lungo le colonne otteniamo  \[\dfrac{\left\{\begin{array}{l} {aa'x+ba'y=a'c} \\ {-aa'x-ab'y=-ac'} \end{array}\right. }{\left(a'b-ab'\right)y=a'c-ac'} \] da cui \[y=\dfrac{ac'-a'c}{ab'-a'b} \]  da cui se  \[\dfrac{a'}{a} \ne \dfrac{b'}{b} \] avremo  \[ab'-a'b\ne 0\]  quindi il sistema ha una e una sola soluzione.
Se \[\dfrac{a'}{a} =\dfrac{b'}{b} \]  avremo due casi  \[\dfrac{a'}{a} =\dfrac{b'}{b} \ne \dfrac{c'}{c} \]  e  \[\dfrac{a'}{a} =\dfrac{b'}{b} =\dfrac{c'}{c} \]
Nel primo caso il sistema è impossibile perché  $ab'-a'b=0$, $a'c-ac'\ne 0$, $b'c-bc'\ne 0$  e quindi poiché è possibile riscrivere il sistema canonico nella forma  \[\left\{\begin{array}{l} {\left(ab'-ab'\right)x=b'c-bc'} \\ {\left(ab'-ab'\right)y=ac'-a'c} \end{array}\right. \]  da cui  \[\left\{\begin{array}{l} {0=b'c-bc'} \\ {0=ac'-a'c} \end{array}\right. \] impossibile.
Nel secondo caso il sistema è indeterminato perché  $ab'-a'b=0$, $a'c-ac'=0$, $b'c-bc'=0$  e quindi giacché è possibile riscrivere il sistema canonico nella forma  \[\left\{\begin{array}{l} {\left(ab'-ab'\right)x=b'c-bc'} \\ {\left(ab'-ab'\right)y=ac'-a'c} \end{array}\right. \]  da cui \[\left\{\begin{array}{l} {0=0} \\ {0=0} \end{array}\right. \]  indeterminato
\section{Metodi di risoluzione}
\label{sec:MetodiDiRisoluzione}
\subsection{Sostituzione}
\label{sec:Sostituzione}
Un sistema lineare in forma canonica si risolve con il metodo di sostituzione\index{Sistema!metodo!sostituzione} isolando una variabile in una equazione e sostituendola nelle altre. 
\begin{esempiot}{}{}
Supponiamo di avere un sistema lineare in forma normale
\[
\begin{cases}
	x+2y=1\\
	3x-y=2
\end{cases}
\]
\end{esempiot}
risolvo la prima rispetto alla x e ottengo 
\[
\begin{cases}
	x=1-2y\\
	3x-y=2
\end{cases}
\]
sostituisco nella seconda ed ottengo
\[
\begin{cases}
	x=1-2y\\
	3(1-2y)-y=2
\end{cases}
\begin{cases}
	x=1-2y\\
	3-6y-y=2
\end{cases}
\begin{cases}
	x=1-2y\\
	-7y=2-3
\end{cases}
\begin{cases}
	x=1-2y\\
	-7y=-1
\end{cases}
\begin{cases}
	x=1-2y\\
	y=\dfrac{1}{7}
\end{cases}
\begin{cases}
	x=1-2\dfrac{1}{7}\\
	y=\dfrac{1}{7}
\end{cases}
\]
\[
\begin{cases}
	x=\dfrac{7-2}{7} \\
	y=\dfrac{1}{7}
\end{cases}
\begin{cases}
	x=\dfrac{5}{7}\\
	y=\dfrac{1}{7}
\end{cases}
\]
\subsection{Confronto}
\label{sec:Confronto}
Un sistema lineare in forma canonica si risolve con il metodo del confronto\index{Sistema!metodo!confronto} risolvendo due equazioni rispetto ad un stessa variabile e confrontando i valori attenuti.
\begin{esempiot}{}{}
supponiamo di avere un sistema lineare in forma normale
\[
\begin{cases}
	x+2y=1\\
	3x-y=2
\end{cases}
\]
\end{esempiot}
Risolvo rispetto ad x e ottengo
\[
\begin{cases}
	x=-2y+1\\
	x=\dfrac{2+y}{3}
\end{cases}
\begin{cases}
	-2y+1=\dfrac{2+y}{3}\\
		x=-2y+1
\end{cases}
\begin{cases}
	-6y+3=2+y\\
	x=-2y+1
\end{cases}
\begin{cases}
	-7y=-1\\
	x=-2y+1
\end{cases}
\begin{cases}
	y=\dfrac{1}{7}\\
	x=-2y+1
\end{cases}
\]
\[
\begin{cases}
	y=\dfrac{1}{7}\\
	x=-2\dfrac{1}{7}+1
\end{cases}
\begin{cases}
	y=\dfrac{1}{7}\\
	x=\dfrac{5}{7}
\end{cases}
\]

\subsection{Riduzione}
\label{sec:Riduzionesist}
Questo metodo è chiamato anche di somma sottrazione e consiste nel sommare o sottrarre le equazioni in modo che vengano determinate le incognite. Iniziamo con un po di vocabolario, chiamiamo riga un'equazione del sistema. Mentre la colonna è formate in verticale dalla stessa incognita nelle varie equazioni.
\begin{esempiot} {}{}
	Risolvere il sistema \[
	\begin{cases}
	2y+3x=2\\
	x+y=5
	\end{cases}\]
\end{esempiot}
Il sistema è formato da due equazioni, quindi da due righe. Il sistema non è ordinato per colonne quindi bisogna riscriverlo in questo modo \[
\begin{cases}
3x+2y=2\\
x+y=5
\end{cases}\]
Il metodo consiste nel moltiplicare le righe per dei valori  opportuni e  sommare o sottrarre lungo le colonne in modo che una variabile scompaia.
\[
\begin{cases}
3x+2y=2\\
\tikzmark{1}3x+3y=15\tikzmark{2}
\end{cases}\]
\begin{tikzpicture}[remember picture, overlay]
\node[below=0.1cm and 0cm of 1](3){};
\node[below=0.1cm and 0cm of 2](4){};
\node[left=2cm and 0cm of 1](5){$3$};
\node[below right=0.2cm and 0.1cm of 5](6) {$0-y=-13$};
\draw (3) edge  (4);
\end{tikzpicture}\\

Moltiplicando la seconda riga per tre e sottraendola alla prima otteniamo che $y=13$. 

\[
\begin{cases}
3x+2y=2\\
\tikzmark{1}2x+2y=10\tikzmark{2}
\end{cases}\]
\begin{tikzpicture}[remember picture, overlay]
\node[below=0.1cm and 0cm of 1](3){};
\node[below=0.1cm and 0cm of 2](4){};
\node[left=2cm and 0cm of 1](5){$2$};
\node[below right=0.2cm and 0.1cm of 5](6) {$x+0 =-8$};
\draw (3) edge  (4);
\end{tikzpicture}\\

Ripartendo dal sistema iniziale e moltiplicando la seconda riga per due otteniamo che $x=-8$

     