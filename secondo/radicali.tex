\chapter{Radicali}
\label{Radicaliradici}
\section{Glossario}Iniziamo con il definire un glossario dei termini che saranno utilizzati nel resto del testo.
\begin{table}[H]
\centering
$\sqrt[n]{a^m}=b$
\begin{itemize}
\item $\sqrt[n]{a^m}$ Radicale
\item n Indice del radicale
\item $a^m$ Radicando
\item m Esponente o potenza del radicando
\item b Radice
\end{itemize}
\caption{Glossario}
\label{tab:RadicaliGlossario}
\end{table}
\begin{table}[H]
\centering
$
\begin{array}{rccc}
\toprule
 &\text{Indice} & \text{Potenza} & \text{Radicando} \\ 
 \midrule
 \sqrt[3]{a}& 3 &1  & a \\ 
 \sqrt[4]{a^3b}& 4 &1  & a^3b \\ 
 \sqrt[4]{a^2}& 4 &2  & a^2 \\
\sqrt{a^5}& 2 &5 & a^5 \\ 
\sqrt[2]{a^5}& 2 &5 & a^5 \\ 
\sqrt[4]{\dfrac{\left( a+b\right)^2 }{c}}& 4 &1 &\dfrac{\left( a+b\right)^2 }{c} \\
\sqrt[3]{a^2+b}& 3 &1 & a^2+b\\
\bottomrule	
\end{array}
$ 
\label{tab:esempiglossario}
\caption{Esempi Glossario}
\end{table}
Iniziamo con il definire il concetto di radice, vale la seguente definizione:
\begin{definizionet}{Radice}{}
	\begin{align*}
	\intertext{Se l'indice  $n$ è pari}
	\sqrt[n]{a}=&b&\Longleftrightarrow&&b^n=a\qquad a\geq 0\quad  n\in\Ni-\lbrace0\rbrace\\
	\intertext{Se l'indice  $n$ è dispari}
	\sqrt[n]{a}=&b&\Longleftrightarrow&&b^n=a\qquad n\in\Ni-\lbrace0\rbrace\\
	\end{align*}
\end{definizionet}
La definizione precedente mostra un legame tra la radice e l'elevamento a potenza. 
\begin{esempiot}{N pari e dispari}{}
\[\sqrt[2]{4}=2\]infatti abbiamo che \[2^2=4\]
Tuttavia non esiste la radice quadrata di indice pari di numeri negativi. Supponiamo che esista, allora \[\sqrt[2]{-4}=x\] Ma per definzione deve accadere che \[x^2=-4\]
Ma il lato sinistro dell'uguaglianza è una quantità positiva mentre il lato destro è negativo quindi, l'uguaglianza non può essere vera. Non esiste la radice quadrata di meno quattro.

Caso diverso sono le radici di indice dispari. Infatti \[\sqrt[3]{-8}=-2\]Infatti \[(-2)^3=-8\]
Quindi esiste la radice cubica di un numero negativo.
\end{esempiot}
%\begin{table}[H]
%\centering
%\begin{itemize}
%%	\item $\sqrt[1]{a}=a\; a\geq 0\forall\; n\in\Ni-\lbrace0\rbrace$
%%	\item $\sqrt[n]{a^n}=a\; a\geq 0\forall\; n\in\Ni-\lbrace 0\rbrace$
%%	\item $\left(\sqrt[n]{a}\right)^n=a\; a\geq 0\forall\; n\in\Ni-\lbrace 0\rbrace$
%%	\item $\sqrt[nk]{a^{mk}}=\sqrt[n]{a^m}\;  a\geq 0\forall\; n,m,k\in\Ni-\lbrace 0\rbrace$\label{Rad:invariantiva}\index{Radicali!proprietà invariantiva}
%\end{itemize}
%\label{tab:propRadicli}
%\caption{Proprietà dei radicali}
%\end{table}
\section{Proprietà dei radicali}
Ponendo che le radici esistano allora:
\subsection{Indice unitario}
\begin{proprietat}{Indice unitario}{}
La radice prima di un numero è il radicando.
\end{proprietat}
\[\sqrt[1]{a}=a\quad a\geq 0\quad\forall n\in\Ni-\lbrace0\rbrace\]
\subsection{Radice elevata all'indice}
\begin{proprietat}{Radice elevata all'indice}{}
	La radice elevata all'indice è il radicando.
\end{proprietat}
\[\left(\sqrt[n]{a}\right)^n=a\quad a\geq 0\quad\forall n\in\Ni-\lbrace 0\rbrace\]
\subsection{Indice e potenza uguali}
\begin{proprietat}{Indice e potenza uguali}{}
	Se l'indice e la potenza sono uguali, la radice è il radicando
\end{proprietat}
\[\sqrt[n]{a^n}=a\quad a\geq 0\forall\quad n\in\Ni-\lbrace 0\rbrace \]
\subsection{Proprietà invariantiva}
\label{Sec:Propinvariantivaradicali}
\begin{proprietat}{Invariantiva}{}
Si ottiene una radice equivalente moltiplicando o dividendo la potenza e l'indice per lo stesso valore diverso da zero.
\end{proprietat}
\[\sqrt[nk]{a^{mk}}=\sqrt[n]{a^m}\;  a\geq 0\forall\; n,m,k\in\Ni-\lbrace 0\rbrace\]
\subsection{Riduzione allo stesso indice}
\label{sec:RiduzioneAlloStessoIndice}

La Proprietà invariantiva permette di ridurre due radici allo stesso indice. Procediamo in questo modo
\begin{enumerate}
	\item Calcolo in m.c.m fra gli indici di tutte le radici
	\item Scrivo delle nuove radici di indice uguale al m.c.m. e per ogni radice
	\begin{itemize}
	\item Divido il m.c.m per l'indice  della radice e moltiplico il numero ottenuto per l'esponente m del radicando della  radice.
	\end{itemize}
\end{enumerate}

\begin{table}[H]
\centering
$
\begin{array}{cccl}
\toprule
\text{Passo} &  &  & \text{Note} \\  
\midrule
0 & \sqrt[3]{a} &\sqrt[5]{b}  &\text{Inizio} \\ 
1 & 15\div 3=5 &15\div 5=3  & mcm(3,5)=15 \\  
2 &\sqrt[15]{a^{1\cdot5}}=\sqrt[15]{a^{5}}  &\sqrt[15]{b^{1\cdot3}}=\sqrt[15]{b^{1\cdot3}}  &\text{Fine} \\
\bottomrule	
\end{array} 
$
\label{tab:Es1Ridstessoindice}
\caption{Esempio riduzione stesso indice}
\end{table}
\begin{table}[H]
\centering
$
\begin{array}{cccl}
\toprule
\text{Passo} &  &  &\text{Note} \\  
\midrule
0 & \sqrt{ab^2} &\sqrt[5]{a+b^2}  &\text{Inizio} \\ 
1 & 10\div 2=5 &10\div 5=2  & mcm(2,5)=10 \\  
2 &\sqrt[10]{a^{1\cdot 5}b^{2\cdot 5}}=\sqrt[10]{a^{2}b^{10}}& \sqrt[10]{\left( a+b^2\right)^{1\cdot 2} }=\sqrt[10]{\left( a+b^2\right)^{2} }   &\text{Fine} \\
\bottomrule	
\end{array} 
$
\label{tab:Es1Ridstessoindice2}
\caption{Esempio riduzione stesso indice}
\end{table}
\subsection{Ordinamento fra radici}
\label{sec:OrdinamentoFraRadici}
Il procedimento di riduzione allo stesso indice permette di confrontare due radici di indice diverso. Basta, dopo aver ridotto le radici allo stesso indice, confrontare i radicandi.
\begin{table}[H]
\centering
$
\begin{array}{ccccl}
\toprule
\text{Passo} &  &  &  &\text{Note} \\ 
\midrule
0 & \sqrt{5} &  &\sqrt[3]{6} & \text{Inizio} \\ 
1 & \sqrt[6]{5^3} &  &\sqrt[6]{6^2}  &\text{Riduzione stesso indice} \\ 
2 & 5^3=125 & > & 6^2=36 & \text{Confronto fra radicandi} \\ 
3 & \sqrt{5} & > & \sqrt[3]{6} & \text{Fine} \\
\bottomrule	
\end{array} 
$
\label{tab:confrontoradicali}
\caption{Esempio confronto radicali}
\end{table}

\subsection{Semplificare radici}
\label{sec:RadiciIriducibili}
La proprietà invariantiva permette di semplificare l'indice e l'esponente  di una radice.
Una radice è irriducibile se non è possibile semplificare l'indice con l'esponente\footnote{Cioè quando l'indice e l'esponente sono primi fra loro}
\begin{table}[H]
\centering
$
\begin{array}{ccl}
\toprule
\text{Passo} &  & \text{Note} \\ 
\midrule
0 &\sqrt[10]{a^2b^4} & \text{Inizio} \\ 
1 &  &  10,2,4 \text{ si dividono per }2\\ 
2 &\sqrt[5]{ab^2}  &\text{Fine}\\
\bottomrule
\end{array} 
$
\label{Tab:radiceriducibile}
\caption{Esempio radice riducibile}
\end{table}
\begin{table}[H]
\centering
$
\begin{array}{ccl}
\toprule
\text{Passo} &  & \text{Note} \\ 
\midrule
0 &\sqrt[10]{a^2+b^4} & \text{Inizio} \\ 
1 &  &  10,2,4 \text{ si dividono per }2 \text{ ma è una somma}\\ 
2 &\sqrt[10]{a^2+b^4}  &\text{Fine}\\
\bottomrule
\end{array} 
$
\label{Tab:radiceriducibilece}
\caption{Esempio radice irriducibile}
\end{table}
\begin{table}[H]
\centering
$
\begin{array}{ccl}
\toprule
\text{Passo} &  & \text{Note} \\ 
\midrule
0 &\sqrt[20]{\dfrac{a^5\left( a^5+b\right)^{10} }{x^{15}}} & \text{Inizio} \\ 
1 &  &  20,10,5,15 \text{ si dividono per }5 \\ 
2 &\sqrt[4]{\dfrac{a\left( a^5+b\right)^{2} }{x^{3}}} & \text{Fine} \\ 
\bottomrule
\end{array} 
$
\label{Tab:radiceriducibilece2}
\caption{Esempio radice riducibile}
\end{table}
\begin{table}[H]
\centering
$
\begin{array}{ccl}
\toprule
\text{Passo} &  & \text{Note} \\ 
\midrule
0 &\sqrt[10]{a^{20}}= & \text{Inizio} \\ 
1 &  &  20,10 \text{ si dividono per }10 \\ 
2 &=\sqrt[1]{a^2}=a^2 & \text{Fine} \\ 
\bottomrule
\end{array} 
$
\label{Tab:radiceriducibilece3}
\caption{Esempio radice riducibile}
\end{table}
\begin{table}[H]
\centering
$
\begin{array}{ccl}
\toprule
\text{Passo} &  & \text{Note} \\ 
\midrule
0 &\sqrt[8]{(-2)^{20}}= & \text{Inizio} \\ 
1 &  &  8,10 \text{ si dividono per }2\\
& \sqrt[4]{(-2)^{5}}&\text{ ma il radicando è negativo, quindi non esiste radice reale} \\ 
 & & \text{Fine} \\ 
\bottomrule
\end{array} 
$
\label{Tab:radiceriducibilece4}
\caption{Esempio radice non riducibile}
\end{table}
\begin{table}[H]
\centering
$
\begin{array}{ccl}
\toprule
\text{Passo} &  & \text{Note} \\ 
\midrule
0 &\sqrt[10]{(-2)^{6}}= & \text{Inizio} \\ 
1 &  &  10,6 \text{ si dividono per }2\\
& \sqrt[5]{(-2)^{3}}&\text{il radicando è negativo ma l'indice è dispari quindi esiste la radice reale} \\ 
2 &=\sqrt[5]{(-2)^{3}} & \text{Fine} \\ 
\bottomrule
\end{array} 
$
\label{Tab:radiceriducibilece5}
\caption{Esempio radice riducibile}
\end{table}
\section{Operazioni con le Radici}
\label{sec:operazioniradici}
\subsection{Prodotto}
\label{sec:prodottoradici}
Il prodotto di due o più radici, con lo stesso indice, è una radice che ha per indice lo stesso indice e per radicando il prodotto dei radicandi.

\[\sqrt[n]{a}\cdot\sqrt[n]{b}=\sqrt[n]{a\cdot b}\]
\begin{table}[H]
\centering
$
\begin{array}{ccl}
\toprule
\text{Passo} &  & \text{Note} \\ 
\midrule
0 &\sqrt[3]{a^3b}\cdot\sqrt[3]{a^4b^2x}=  & \text{Inizio} \\ 
1 &  &  \text{i due radicali hanno lo stesso indice} \\ 
2 &=\sqrt[3]{a^3b}\cdot\sqrt[3]{a^4b^2x}=\sqrt[3]{a^7b^3x} & \text{Fine} \\ 
\bottomrule
\end{array} 
$
\label{Tab:radicprodotto1}
\caption{Esempio prodotto di radici con lo stesso indice}
\end{table}
Se le radici hanno indice diverso bisogna prima ridurle allo stesso indice.
\begin{table}[H]
\centering
$
\begin{array}{ccl}
\toprule
\text{Passo} &  & \text{Note} \\ 
\midrule
0 &\sqrt[3]{a^3b}\cdot\sqrt[5]{a^4b^2x}=  & \text{Inizio} \\ 
1 &  &   \text{I due radicali hanno indice diverso} \\ 
2 &=\sqrt[15]{a^{15}b^5}\cdot\sqrt[15]{a^4b^2x}= & \text{Riduco allo stesso indice} \\ 
3 &=\sqrt[15]{a^{15}b^5}\cdot\sqrt[15]{a^4b^2x}=\sqrt[15]{a^{17}b^{11}x^3} & \text{Fine} \\ 
\bottomrule
\end{array} 
$
\label{Tab:radiceprodotto2}
\caption{Esempio prodotto di radici con indice diverso}
\end{table}
\begin{table}[H]
\centering
$
\begin{array}{ccl}
\toprule
\text{Passo} &  & \text{Note} \\ 
\midrule
0 &\sqrt[3]{\dfrac{ya}{x}}\cdot\sqrt{\dfrac{x^2}{y}}\cdot\sqrt[6]{\dfrac{y}{x^4}}  & \text{Inizio} \\ 
1 &  &   \text{I due radicali hanno indice diverso} \\ 
2 &=\sqrt[15]{a^{15}b^5}\cdot\sqrt[15]{a^4b^2x}= & \text{Riduco allo stesso indice} \\ 
3 &=\sqrt[15]{a^{15}b^5}\cdot\sqrt[15]{a^4b^2x}=\sqrt[15]{a^{17}b^{11}x^3} & \text{Fine} \\ 
\bottomrule
\end{array} 
$
\label{Tab:radiceprodotto3}
\caption{Esempio prodotto di radici con indice diverso}
\end{table}
\subsubsection{Trasporto di un termine fuori del segno di radice}
\label{sec:Trasportofuoriradici}
Se la potenza m del radicando è maggiore o uguale all'indice n del radicando allora:
\[\sqrt[n]{a^m}=a^q\sqrt[n]{a^r}\] 
\begin{itemize}
\item n Indice radice
\item m Esponente o potenza del radicando
\item q Quoziente della divisione di m per n
\item r Resto della divisione di m per n\footnote{Per ottenere r basta moltiplicare la parte decimale della divisione di per n. Es $3\div 2=1,5$ $r=0,5\cdot 2=1$ }
\end{itemize}
\begin{table}[H]
\centering
$
\begin{array}{ccc}
\toprule
\text{Passo} &  & \text{Note} \\
\midrule
0& \sqrt[3]{32}=\sqrt[3]{2^5} & \text{Inizio} \\ 
1 & \text{Divido } 5 \text{  per } 3 &q=1\text{  }r=2  \\ 
2&  \sqrt[3]{32}=\sqrt[3]{2^5}=2^1\sqrt[3]{2^2}=2\sqrt[3]{4}& \text{Fine} \\ 
\bottomrule 
\end{array}
$ 
\label{tab:Trasportofuoriradici1}
\caption{Esempio trasporto di un termine fuori del segno di radice}
\end{table}
\begin{table}[H]
\centering
$
\begin{array}{ccc}
\toprule
\text{Passo} &  & \text{Note} \\
\midrule
0& \sqrt[8]{a^{15}} & \text{Inizio} \\ 
1 & \text{Divido } 15 \text{  per } 8 &q=1\text{  }r=7  \\ 
2&  \sqrt[8]{a^{15}}=a^1\sqrt[8]{a^7}=a\sqrt[8]{a^7}& \text{Fine} \\ 
\bottomrule 
\end{array}
$ 
\label{tab:Trasportofuoriradici2}
\caption{Esempio trasporto di un termine fuori del segno di radice}
\end{table}
\begin{table}[H]
\centering
$
\begin{array}{ccc}
\toprule
\text{Passo} &  & \text{Note} \\
\midrule
0& \sqrt{72}=\sqrt{3^2\cdot 2^3} & \text{Inizio} \\ 
1 & \text{Divido } 2 \text{  per } 2 q=1\text{  }r=0 \\
&\text{ Divido } 3 \text{  per } 2\text{  } q=1\text{  }r=1    \\ 
2&  \sqrt{72}=\sqrt{3^2\cdot 2^3}=3^1\cdot 2^1\sqrt{3^0 2}=6\sqrt{2}& \text{Fine} \\ 
\bottomrule 
\end{array}
$ 
\label{tab:Trasportofuoriradici3}
\caption{Esempio trasporto di un termine fuori del segno di radice}
\end{table}
\begin{table}
\centering
$
\begin{array}{ccc}
\toprule
\text{Passo} &  & \text{Note} \\
\midrule
0& \sqrt[3]{\dfrac{a^6b^2}{c^4\left( a+b\right)^3 }} & \text{Inizio} \\ %
1 & considero gli esponenti maggiori o uguali all'indice della radice cioè 6,4,3 \\
&\text{ Divido } 6 \text{  per } 3\text{  } q=2\text{  }r=0    \\ 
&\text{ Divido } 4 \text{  per } 3\text{  } q=1\text{  }r=1    \\
&\text{ Divido } 3 \text{  per } 3\text{  } q=1\text{  }r=0    \\
2& \sqrt[3]{\dfrac{a^6b^2}{c^4\left( a+b\right)^3 }}=\dfrac{a^2}{c\left(a+b\right)}\sqrt[3]{\dfrac{a^0b^2}{c^1\left( a+b\right)^0 }} =\dfrac{a^2}{c\left(a+b\right)  }\sqrt[3]{\dfrac{b^2}{c}} & \text{Fine} \\ 
\bottomrule 
\end{array}
$ 
\label{tab:Trasportofuoriradici4}
\caption{Esempio trasporto di un termine fuori del segno di radice}
\end{table}
\subsubsection{Trasporto di un termine dentro il segno di radice}
\label{sec:Trasportodentroradici}
\[b^p\sqrt[n]{a^m}=\sqrt[n]{b^{p\cdot n}a^m}\]
\begin{table}[H]
\centering
$
\begin{array}{ccc}
\toprule
\text{Passo} &  & \text{Note} \\
\midrule
0& 3\sqrt[2]{2} & \text{Inizio} \\ 
1& 3\sqrt[2]{2}=\sqrt{3^2\cdot 2}=\sqrt[2]{18} & \text{Fine} \\ 
\bottomrule 
\end{array}
$ 
\label{tab:Trasportodentroradici1}
\caption{Esempio trasporto di un termine dentro il segno di radice}
\end{table}
\begin{table}[H]
\centering
$
\begin{array}{ccc}
\toprule
\text{Passo} &  & \text{Note} \\
\midrule
0& \dfrac{c}{\left( a+b\right)^2 }\sqrt[3]{c} & \text{Inizio} \\ 
1& \dfrac{c}{\left( a+b\right)^2 }\sqrt[3]{c}=\sqrt[3]{\dfrac{c^{1\cdot 3}c}{\left(a+b \right) }^{2\cdot 3}}=\sqrt[3]{\dfrac{c^{4}}{\left(a+b \right) }^{6}} & \text{Fine} \\ 
\bottomrule 
\end{array}
$ 
\label{tab:Trasportodentroradici2}
\caption{Esempio trasporto di un termine dentro il segno di radice}
\end{table}
\section{Potenze}
\label{sec:PotenzeRadici}
\[\left( \sqrt[n]{a}\right)^m=\sqrt[n]{a^m}\]
\begin{table}[H]
\centering
$
\begin{array}{cc}
\toprule
\left( \sqrt[4]{a}\right)^3=\sqrt[4]{a^3} \\ 
\left( \sqrt[3]{a+b}\right)^2=\sqrt[2]{\left( a+b\right)^2 } \\ 
\left( \sqrt[5]{a^2b}\right)^3=\sqrt[4]{a^6b^3} \\ 
\bottomrule 
\end{array}
$ 
\label{tab:potenzeradici1}
\caption{Esempi potenze radicali}
\end{table}
\section{Quoziente} 
\label{sec:quozienteradicali}
\[\dfrac{\sqrt[n]{a}}{\sqrt[n]{b}}=\sqrt[n]{\dfrac{a}{b}}\]
\section{Radice di radice}
\label{sec:radicediradice}
\[\sqrt[n]{\sqrt[p]{a}}=\sqrt[n\cdot p]{a}\]
\section{Somma}
\label{sec:SommaReali}
\begin{itemize}
\item Due radici sono simile se hanno lo stesso indice e  lo stesso radicando.
\item Se consideriamo una radice come un monomio avremo che il numero che compare davanti al simbolo di radice è la "`parte numerica' mentre la radice è la "`parte letterale'. Possiamo dare la seguente definizione: La somma di due radici simili è una radice simile alle precedenti che ha per parte numerica la somma algebrica delle parti numeriche.
\end{itemize}
\section{Razionalizzazione del denominatore}
\label{sec:razzionalizzazionedenominatoreradici}
\subsection{Primo caso}
\bassapriorita{Inserire razionalizzazioni}
\label{sec:razionzinalizzadioneden1caso}
\subsection{Secondo caso}
\label{sec:razionzinalizzadioneden2caso}


