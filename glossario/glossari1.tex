%a
\newglossaryentry{opaddg}{name={Addizione},
description={L'addizione  operazione binaria}}
\newglossaryentry{addendog}{name={Addendo},
description={termine dell'addizione}}
\newglossaryentry{albbing}{name={Albero binario},
description={Un albero binario è formato da un nodo detto radice da cui si staccano due nodi detti figli. Un nodo senza figli è detto foglia}}

%D
\newglossaryentry{opdivfg}{name={Divisione},
description={Divisione  operazione binaria}}
\newglossaryentry{differenzag}{name={Differenza},
description={risultato sottrazione}}
\newglossaryentry{dividendog}{name={Dividendo},
description={primo termine divisione}}
\newglossaryentry{divisoreg}{name={Divisore},
description={secondo termine divisione}}
%E

%F
\newglossaryentry{fattoreg}{name={Fattore},
description={termine della moltiplicazione}}
\newglossaryentry{frazpropg}{name={Frazione propria},
description={nella frazione propria il numeratore è minore del denominatore}}
\newglossaryentry{frazinpropg}{name={Frazione impropria},
description={nella frazione impropria il numeratore è maggiore del denominatore}}
\newglossaryentry{frazineappag}{name={Frazione apparente},
description={nella frazione apparente il numeratore è multiplo del denominatore}}
\newglossaryentry{frazionegeng}{name={Frazione generatrice},
description={la frazione corrispondente ad un numero decimale dato}}
\newglossaryentry{frazioniequivg}{name={Frazioni equivalenti},
description={due frazioni sono equivalenti quando rappresentano lo stesso quoziente.}}
%I
\newglossaryentry{insdisg}{name={Insieme discreto},
description={un insieme è discreto nel senso che fra un numero e il suo successivo non vi è nessun altro elemento dell'insieme}}
%M
\newglossaryentry{minuendog}{name={Minuendo},
description={primo termine sottrazione}}
\newglossaryentry{opdifg}{name={Moltiplicazione},
description={operazione binaria}}
\newglossaryentry{multiplog}{name={Multiplo},
description={un numero $a$ è multiplo di un altro numero $b$ se esiste un numero $c$ tale che $a=b\cdot c$ }}
%N
\newglossaryentry{numng}{name={Numeri naturali},
description={insieme numerico $\Ni=\Set{0,1,2,3,\dots,}$}}
\newglossaryentry{numnprimifralorog}{name={Numeri primi fra loro},
description={due numeri sono primi fra loro se l'unico numero che li divide entrambi è uno}}	
\newglossaryentry{numerodecimaleg}{name={Numero decimale},
description={numero formato da due parti separate dalla virgola chiamate parte intera e parte decimale}}
\newglossaryentry{numerodecimalfinitog}{name={Numero decimale finito},
description={numero decimale con la parte decimale composta da un numero finito di cifre},see={numerodecimaleg}}
\newglossaryentry{numerodecimalinfinitog}{name={Numero decimale infinito},
description={Numero decimale con la parte decimale composta da un numero infinito di cifre},see={numerodecimaleg}}	
\newglossaryentry{numerodecimalinfinitoperg}{name={Numero decimale periodico},
description={numero decimale con la parte decimale composta da un numero finito di cifre, periodo, che si ripetono all'infinito},see={numerodecimaleg}}
\newglossaryentry{numerodecimalinfinitopermistg}{name={Numero decimale periodico misto},
description={numero decimale con la parte decimale divisa in una parte finita detta antiperiodo e  da un numero finito di cifre, periodo, che si ripetono all'infinito},see={numerodecimaleg}}
\newglossaryentry{numerireciprocig}{name={Numeri reciproci},
description={due numeri sono reciproci se il loro prodotto è uno}}
%P
\newglossaryentry{potenzag}{name={Potenza},
description={Potenza operazione binaria}}
\newglossaryentry{primog}{name={Primo},
description={Numero divisibile solo per se stesso e per l'unità}}

\newglossaryentry{prodottog}{name={Prodotto},
description={Risultato moltiplicazione}}
%Q
\newglossaryentry{quozienteg}{name={Quoziente},
description={Risultato divisione}}
%R

%S

\newglossaryentry{opdiffg}{name={Sottrazione},
description={Sottrazione  operazione binaria}}
\newglossaryentry{sommag}{name={Somma},
description={Risultato dell'addizione}}
\newglossaryentry{sottraendog}{name={Sottraendo},
description={Secondo termine sottrazione}}
\newglossaryentry{scompfatprimig}{name={Scomposizione in fattori primi},
description={Scrivere un numero come prodotto di numeri primi}}
\newglossaryentry{frazionesempg}{name={Semplificare una frazione},
description={Semplificare una frazione significa dividere il numeratore e il denominatore per il loro Massimo Comune Divisore ($\mcd$). Per la proprietà invariantiva la frazione ottenuta è equivalente a quella data}}

%T



	