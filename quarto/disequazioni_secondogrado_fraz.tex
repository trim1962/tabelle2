\section{Disequazioni frazionarie di secondo grado}
\begin{definizionet}{Disequazione frazionaria in forma normale}{}
	Una disequazione di secondo grado\index{Disequazione!secondo grado} frazionaria è in forma normale se \[\dfrac{f(x)}{g(x)}\;\begin{cases}
	>\\
	<\\
	\leq\\
	\geq
	\end{cases} 0\; \text{con}\; g(x)\neq 0\]
\end{definizionet}
\begin{osservazionet}{}{}
	Le disequazioni frazionarie di secondo grado in forma normale\index{Disequazione!forma normale} hanno  una delle seguenti forme
	\begin{align*}
	\dfrac{ax^2+bx+c}{a_1x^2+b_1x+c_1}&\quad\begin{cases}
	>\\
	<\\
	\leq\\
	\geq
	\end{cases} 0\\
	\dfrac{ax^2+bx+c}{a_1x+b_1}&\quad\begin{cases}
	>\\
	<\\
	\leq\\
	\geq
	\end{cases} 0\\
	\dfrac{ax+b}{a_1x^2+b_1x+c_1}&\quad\begin{cases}
	>\\
	<\\
	\leq\\
	\geq
	\end{cases} 0\\
	\end{align*}
	\end{osservazionet} 
\begin{osservazionet}{}{}
\begin{align*}
\intertext{Sono disequazioni frazionarie di secondo grado in forma normale}
&\dfrac{x^2+3x+4}{x+2}\geq 0\\
&\dfrac{x^2+3x+5}{x^2-3x+1}<0\\
&\dfrac{x+3}{x^2-2x+1}>0\\
\intertext{non è una disequazione frazionaria di secondo grado la frazione}
&\dfrac{x^2+3x+2}{3}<0\\
\end{align*}
\end{osservazionet}

Risolvere una disequazione frazionaria di secondo grado equivale a trovare il segno del rapporto fra il numeratore e il denominatore della frazione. In pratica bisognerà risolvere due disequazioni e confrontare fra loro i grafici utilizzando la regola dei segni come nell'\cref{exa:frazionaria1}

\begin{esempiot}{Disequazione frazionaria uno}{frazionaria1} 
	Consideriamo la disequazione \[\dfrac{x^2+2x+1}{3x^2+5x+2}\leq 0 \]
	Calcolo i segno della frazione iniziando dal segno  del numeratore.
	\begin{align*}
	&x^2+2x+1\geq 0
	\intertext{ad essa è associata l'equazione}
	&x^2+2x+1=0\\
	&x_{1,2}=\dfrac{-2\pm\sqrt{4-4}}{2}=\dfrac{-2}{2}=-1
	\end{align*}
	$\Delta=0$ $a>0$ quindi il procedimento è analogo a quello  dell'\cref{exa:DeltaUgualeaZeroaMaggiore} otteniamo il grafico~\cref{fig:DeltaUgualeaZeroGraficoEsempioDF1}
	
	Passo al denominatore e risolvo
		\begin{align*}
	&3x^2+5x+2>0
	\intertext{ad essa è associata l'equazione}
	&3x^2+5x+2=0\\
	&x_{1,2}=\dfrac{-5\pm\sqrt{25-24}}{6}=\dfrac{-5\pm1}{6}=
	\begin{cases}
	x_1=-1\\x_2=-\dfrac{2}{3}
	\end{cases}
	\end{align*}
	 	$\Delta>0$ $a>0$ quindi il procedendo in modo analogo a quello  dell'\cref{exa:DeltaMaggiorediZeroamaggiore} otteniamo il  grafico~\cref{fig:DeltaUgualeaZeroGraficoEsempioDF2} di tipo DICE. Sovrapponendo i due grafici otteniamo il grafico~\cref{fig:DFesempio1}. La  disequazione chiede quando la frazione è negativa quindi la risposta è\[-\dfrac{3}{2}<x<-1 \] Non sarà mai uguale a zero dato che gli zeri appartengono al denominatore della frazione.
\end{esempiot}
\begin{figure}
	\centering
	\includestandalone[width=8.5cm]{quarto/DisSecGrado/parabolaDeltazeroApiuGrafico3}
	\caption{Segno $\Delta=0$ $a>0$}
	\label{fig:DeltaUgualeaZeroGraficoEsempioDF1}
\end{figure}
\begin{figure}
	\centering
	\includestandalone[width=8.5cm]{quarto/DisSecGrado/parabolaDeltapiuApiuGrafico3}
	\caption{Segno $\Delta>0$ $a>0$}
	\label{fig:DeltaUgualeaZeroGraficoEsempioDF2}
\end{figure}
\begin{figure}
	\centering
	\includestandalone[width=8.5cm]{quarto/DisSecGrado/DFesempio1}
	\caption{Segno disequazione frazionaria}
	\label{fig:DFesempio1}
\end{figure}
\begin{esempiot}{Disequazione frazionaria due}{frazionaria2} 
	Consideriamo la disequazione \[\dfrac{1-3x}{7x-6x^2-2}\leq 0 \]
Calcolo i segno della frazione iniziando dal segno  del numeratore.
\begin{align*}
&1-3x\geq0\\
&-3x\geq-1\\
&x\leq\dfrac{1}{3}
\end{align*}
Otteniamo un grafico come il grafico~\vref{fig:primogradopiumeno}.

Resta da determinare il segno del denominatore 
\begin{align*}
&7x-6x^2-2>0
\intertext{ad essa è associata l'equazione}
&7x-6x^2-2=0\\
&x_{1,2}=\dfrac{-7\pm\sqrt{49-48}}{-12}=\dfrac{-7\pm1}{-12}=
\begin{cases}
x_1=\dfrac{2}{3}\\
\\x_2=\dfrac{1}{2}
\end{cases}
\end{align*} 
Dato che $\Delta>0$ e $a<0$ il grafico è di tipo DICE, quindi otteniamo il grafico~\vref{fig:DeltaMaggioreZeroGraficoEsempioDF3}. Resta da sovrapporre i due grafici ed ottenere il grafico della frazione.

Si ottiene un grafico come il grafico~\cref{fig:DFesempio2}. Dato che la disequazione chiede quando la frazione sia minore o uguale a zero leggendo il grafico otteniamo la soluzione \[ x\leq \dfrac{1}{3}\] \[\dfrac{1}{2}<x<\dfrac{2}{3} \]
\end{esempiot}
\begin{figure}
	\centering 
	\includestandalone[width=8.5cm]{quarto/DisSecGrado/primogradoPiuMeno}
	\caption{Segno disequazione primo grado}
	\label{fig:primogradopiumeno}
\end{figure}
\begin{figure}
	\centering
\includestandalone[width=8.5cm]{quarto/DisSecGrado/parabolaDeltapiuAmenoGrafico3}
	\caption{Segno $\Delta>0$ $a<0$}
	\label{fig:DeltaMaggioreZeroGraficoEsempioDF3}
\end{figure}
\begin{figure}
	\centering
	\includestandalone[width=8.5cm]{quarto/DisSecGrado/DFesempio2}
	\caption{Segno disequazione frazionaria}
	\label{fig:DFesempio2}
\end{figure}
\begin{esempiot}{Disequazione frazionaria tre}{frazionaria3} 
	Consideriamo la disequazione \[\dfrac{6-x-x^2}{1-x}\geq 0 \]
Calcolo i segno della frazione iniziando dal segno  del numeratore.
\begin{align*}
&6-x-x^2\geq0\\
\intertext{ad essa è associata l'equazione}
&-x^2-x-6=0\\
&x_{1,2}=\dfrac{1\pm\sqrt{1+24}}{-2}=\dfrac{1\pm5}{-2}=
\begin{cases}
x_1=-3\\
\\x_2=2
\end{cases}
\end{align*}
Il grafico della disequazione è quindi DICE ed è quello riportato dalla~\vref{fig:DeltaMaggioreZeroGraficoEsempioDF4}

Determino il segno del denominatore
\begin{align*}
&1-x>0\\
&-x>-1\\
&x<1
\end{align*}
Ottengo il grafico~\vref{fig:primogradopiumeno1}
Resta da sovrapporre i due grafici ed ottenere il grafico della frazione.

Si ottiene un grafico come il grafico~\cref{fig:DFesempio3}. Dato che la disequazione chiede quando la frazione sia maggiore o uguale a zero leggendo il grafico otteniamo la soluzione \[-3\leq x<1\]
\[2\leq x\]
\end{esempiot}
\begin{figure}
	\centering
	\includestandalone[width=8.5cm]{quarto/DisSecGrado/parabolaDeltapiuAmenoGrafico4}
	\caption{Segno $\Delta>0$ $a<0$}
	\label{fig:DeltaMaggioreZeroGraficoEsempioDF4}
\end{figure}
\begin{figure}
	\centering 
	\includestandalone[width=8.5cm]{quarto/DisSecGrado/primogradoPiuMeno2}
	\caption{Segno disequazione primo grado}
	\label{fig:primogradopiumeno1}
\end{figure}
\begin{figure}
	\centering
	\includestandalone[width=8.5cm]{quarto/DisSecGrado/DFesempio3}
	\caption{Segno disequazione frazionaria}
	\label{fig:DFesempio3}
\end{figure}