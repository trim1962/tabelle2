\chapter{Le proporzioni}
\label{sec:LeProporzioni}
\section{Proporzioni semplici}
\label{sec:ProporzioniSemplici}


Una proporzione è un'uguaglianza fra frazioni
\[\dfrac{a}{b}=\dfrac{c}{d}\]
\[medi\index{Proporzione!medi}\]
\[{\underbrace{a:\overbrace{b=c}:d}}\]
\[estremi\index{Proporzione!estremi}\]
\[a,b,c,d\in N\]

Una proporzione con i medi\index{Proporzione!medi} uguali si dice continua\index{Proporzione!continua}

\section{Proprietà delle proporzioni}
\label{sec:ProprietaDelleProporzioni}
\subsection{Propriet\'{a} fondamentale delle proporzioni}\label{prop:fond}
\begin{enumerate}
	\item In una proporzione il prodotto dei medi\index{Proporzione!medi} è uguale al prodotto degli estremi\index{Proporzione!estremi} \[a\cdot d= b\cdot c\]
	\item In una proporzione qualunque un medio incognito è uguale al prodotto degli estremi\index{Proporzione!estremi} fratto l'altro medio. 
	\begin{align*}
	a:x=c:d&\\
	x=\dfrac{a\cdot d}{c}&
	\end{align*}
	\item In una proporzione qualunque un estremo incognito è uguale al prodotto degli medi\index{Proporzione!medi} fratto l'altro medio. 
	\begin{align*}
	x:b=c:d&\\
	x=\dfrac{b\cdot c}{d}&
	\end{align*}
	\item Il medio proporzionale fra due numeri dati à uguale alla radice quadrata del  prodotto degli estremi.
		\begin{align*}
		a:x=x:d&\\
		x=\sqrt{a\cdot d}&
		\end{align*}
	\item In una proporzione la somma dei primi due termini sta al primo (secondo) come dei due restanti termini sta al terzo (quarto)\footnote{dimostriamo la prima uguaglianza:
		\begin{gather*}
		\dfrac{a}{b}=\dfrac{c}{d}\\
		\dfrac{a}{b}+1=\dfrac{c}{d}+1\\
		\dfrac{a+b}{b}=\dfrac{c+d}{d}
		\end{gather*}
		
		dimostriamo la seconda uguaglianza
		\begin{gather*}
		\dfrac{b}{a}=\dfrac{d}{c}\\
		\dfrac{b}{a}+1=\dfrac{d}{c}+1\\
		\dfrac{a+b}{a}=\dfrac{c+d}{c}
		\end{gather*}}
	\[(a+b):b = (c+d):d\]
	\[(a+b):a = (c+d):c\]
	\item In una proporzione la differenza fra il maggiore e il minore dei primi due termini sta al primo (secondo) come la differenza fra il maggiore e il minore dei due restanti termini sta al terzo (quarto)
	\[(a-b):a = (c-d):c\]
	\[(a-b):b = (c-d):d\]
	\item Una proporzione \'{e} ancora una proporzione scambiando fra loro i medi\index{Proporzione!medi} (o gli estremi\index{Proporzione!estremi})
	\item In una proporzione la somma degli antecedenti\index{Proporzione!antecedenti} sta alla somma dei conseguenti\index{Proporzione!conseguenti} come un antecedente sta al suo conseguente\footnote{\begin{gather*}
		\dfrac{c}{a}=\dfrac{d}{b}
		\\1+\dfrac{c}{a}=1+\dfrac{d}{b}
		\\ \dfrac{a+c}{a}=\dfrac{b+d}{b}
		\\ (a+c):a= (b+d):b
		\end{gather*}
		\[(a+c):(b+d)=a:b\]}.
	\[(a+c):(b+d)=a:b\]
	\item In una proporzione, la differenza tra il maggiore e il minore degli antecedenti,\index{Proporzione!antecedenti} sta alla differenza tra il maggiore e il minore dei conseguenti,\index{Proporzione!conseguenti} come un antecedente sta al suo conseguente.
	\[(a-c):(b-d)=a:b\]
\end{enumerate}
\section{Serie di rapporti uguali}
\label{sec:serieDiRapportiUguali}

Una serie di rapporti\index{Rapporti uguali!rapporti} uguali è l'uguaglianza fra tre o più frazioni\label{sec:sRaUguali}

\[
\dfrac{a_{1}}{b_{1}}=\dfrac{a_{2}}{b_{2}}=\cdots=\dfrac{a_{n}}{b_{n}}\qquad n\geq3\]
[Comporre generalizzata]\index{Rapporti uguali!comporre generalizzata}

In una serie di rapporti uguali la somma degli antecedenti sta alla somma dei conseguenti, come un antecedente sta al proprio conseguente\footnote{Dalla definizione\nobs\vref{sec:sRaUguali}
	\begin{gather*}
	\dfrac{a_{2}}{a_{1}}=\dfrac{b_{2}}{b_{1}}\\
	\dfrac{a_{3}}{a_{1}}=\dfrac{b_{3}}{b_{1}}
	\\
	\ldots%
	\\
	\dfrac{a_{n}}{a_{1}}=\dfrac{b_{n}}{b_{1}}
	\end{gather*}
	
	sommando membro a membro ottengo
	\[\dfrac{a_{2}}{a_{1}}+\dfrac{a_{3}}{a_{1}}+\cdots+\dfrac{a_{n}}{a_{1}}=\dfrac{b_{2}}{b_{1}}+\dfrac{b_{3}}{b_{1}}+\cdots+\dfrac{b_{n}}{b_{1}}\]
	da cui
	\[1+\dfrac{a_{2}}{a_{1}}+\dfrac{a_{3}}{a_{1}}+\cdots+\dfrac{a_{n}}{a_{1}}=\dfrac{b_{2}}{b_{1}}+\dfrac{b_{3}}{b_{1}}+\cdots+\dfrac{b_{n}}{b_{1}}+1\]
	da cui 
	\[\dfrac{a_{1}+a_{2}+\cdots+a_{n}}{a_{1}}=\dfrac{b_{1}+b_{1}+\cdots+b_{n}}{b_{1}}\]
	da cui la prima relazione.
	
	Procedendo in maniera analoga otteniamo il resto.}
\begin{gather*}
	(a_{1}+a_{2}+\ldots+a_{n}): (b_{1}+b_{2}+\ldots+b_{n})=a_{1}:b_{1}
	\\(a_{1}+a_{2}+\ldots+a_{n}): (b_{1}+b_{2}+\ldots+b_{n})=a_{2}:b_{2}
	\\\ldots\ldots\ldots\ldots\ldots\ldots%
	\\(a_{1}+a_{2}+\ldots+a_{n}): (b_{1}+b_{2}+\ldots+b_{n})=a_{n}:b_{n}
\end{gather*}