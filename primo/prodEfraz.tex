\chapter{Somme, prodotti e frazioni}
\label{sec:prodottiEDivisioni}
\section{Segni}
\label{sec:segnioperazioni}
\begin{table}[H]
	\begin{subtable}[b]{.5\linewidth}
		\centering
		$
		\begin{array}{lcc}
		\toprule
		operazione&&segno\\
		\midrule
		\bm{(-a)\cdot(-b)}&&+(a)\cdot(b)\\
		\midrule
		\bm{(+a)\cdot(-b)}&&-(a)\cdot(b)\\
		\midrule
		\bm{(-a)\cdot(+b)}&&-(a)\cdot(b)\\
		\midrule
		\bm{(+a)\cdot(+b)}&&+(a)\cdot(b)\\
		\bottomrule
		\end{array}
		$
		\caption{Segno prodotto algebrico}\label{tab:segnoprodottoalgebrico}
	\end{subtable}%
	\begin{subtable}[b]{.5\linewidth}
		\centering
		$
		\begin{array}{lcc}
		\toprule
		operazione&&segno\\
		\midrule
		\bm{(-a)\div(-b)}&&+(a)\div(b)\\
		\midrule
		\bm{(+a)\div(-b)}&&-(a)\div(b)\\
		\midrule
		\bm{(-a)\div(+b)}&&-(a)\div(b)\\
		\midrule
		\bm{(+a)\div(+b)}&&+(a)\div(b)\\
		\bottomrule
		\end{array}
		$
		\caption{Segno divisione algebrica}\label{tab:segnodivisioneoalgebrica}
	\end{subtable}
	\begin{subtable}[b]{\linewidth}
		\centering
		$
		\begin{array}{lcc}
		\toprule
		operazione&&segno\\
		\midrule
		\bm{-a-b}&&-\\
		\midrule
		\multirow{3}*{$\bm{-a+b}$}&\abs{a}>\abs{b}&-\\
		&\abs{a}=\abs{b}&0\\
		&\abs{a}<\abs{b}&+\\
		\midrule
		\multirow{3}*{$\bm{+a-b}$}&\abs{a}>\abs{b}&+\\
		&\abs{a}=\abs{b}&0\\
		&\abs{a}<\abs{b}&-\\
		\midrule
		\bm{+a+b}&&+\\
		\bottomrule
		\end{array}
		$
		\caption{Segno somma algebrica}\label{tab:segnosommaalgebrica}
	\end{subtable}
	\caption{Segni}
	\label{Tab:Segni operazioni}
	
\end{table}
%\begin{table}[H]
%\centering
%
%\subfloat[][Segno prodotto algebrico\label{tab:segnoprodottoalgebrico}]{
%$
%\begin{array}{lcc}
%\toprule
%operazione&&segno\\
%\midrule
%\bm{(-a)\cdot(-b)}&&+(a)\cdot(b)\\
%\midrule
%\bm{(+a)\cdot(-b)}&&-(a)\cdot(b)\\
%\midrule
%\bm{(-a)\cdot(+b)}&&-(a)\cdot(b)\\
%\midrule
%\bm{(+a)\cdot(+b)}&&+(a)\cdot(b)\\
%\bottomrule
%\end{array}
%$
%}\qquad
%\subfloat[][Segno divisione algebrica\label{tab:segnodivisioneoalgebrica}]{
%$
%\begin{array}{lcc}
%\toprule
%operazione&&segno\\
%\midrule
%\bm{(-a)\div(-b)}&&+(a)\div(b)\\
%\midrule
%\bm{(+a)\div(-b)}&&-(a)\div(b)\\
%\midrule
%\bm{(-a)\div(+b)}&&-(a)\div(b)\\
%\midrule
%\bm{(+a)\div(+b)}&&+(a)\div(b)\\
%\bottomrule
%\end{array}
%$
%}\\
%\subfloat[][Segno somma algebrica\label{tab:segnosommaalgebrica}]{
%$
%\begin{array}{lcc}
%\toprule
%operazione&&segno\\
%\midrule
%\bm{-a-b}&&-\\
%\midrule
%\multirow{3}*{$\bm{-a+b}$}&\abs{a}>\abs{b}&-\\
%&\abs{a}=\abs{b}&0\\
%&\abs{a}<\abs{b}&+\\
%\midrule
%\multirow{3}*{$\bm{+a-b}$}&\abs{a}>\abs{b}&+\\
%&\abs{a}=\abs{b}&0\\
%&\abs{a}<\abs{b}&-\\
%\midrule
%\bm{+a+b}&&+\\
%\bottomrule
%\end{array}
%$
%}
%\caption{Segni}
%\label{Tab:Segni operazioni}
%\end{table}
\section{Precedenze}
\label{sec:Precedenze operazioni}
\begin{table}[H]
	\begin{subtable}[b]{.5\linewidth}
		\centering
		\begin{tabular}{cl}
			\toprule
			precedenza&operazione\\
			\midrule
			\phantom{$\Bigl($}1&potenza\\[.4cm]
			\phantom{$\Bigl[$}2& prodotto divisione\\[.4cm]
			\phantom{$\Bigl\{$}3& somma sottrazione\\[.4cm]
			\bottomrule
		\end{tabular}
		\caption{Precedenza operazioni}\label{tab:precedenzaoperazioni}
	\end{subtable}%
	\begin{subtable}[b]{.5\linewidth}
		\centering
		\begin{tabular}{cl}
			\toprule
			precedenza&parentesi\\
			\midrule
			1&$\Bigl(\dots\Bigr)$\\[.4cm]
			2& $\Bigl[\dots\Bigr]$\\[.4cm]
			3& $\Bigl\{\dots\Bigr\}$\\[.4cm]
			\bottomrule
		\end{tabular}
		\caption{Precedenza parentesi}\label{tab:precedenzaparentesi}
	\end{subtable}
	\caption{Precedenze}
	\label{Tab:precedenze}
\end{table}
%\begin{table}[H]
%\centering
%\subfloat[][Precedenza operazioni\label{tab:precedenzaoperazioni}]{
%\begin{tabular}{cl}
%\toprule
%precedenza&operazione\\
%\midrule
%\phantom{$\Bigl($}1&potenza\\[.4cm]
%\phantom{$\Bigl[$}2& prodotto divisione\\[.4cm]
%\phantom{$\Bigl\{$}3& somma sottrazione\\[.4cm]
%\bottomrule
%\end{tabular}
%
%}\qquad
%\subfloat[][Precedenza parentesi\label{tab:precedenzaparentesi}]{
%\begin{tabular}{cl}
%\toprule
%precedenza&parentesi\\
%\midrule
%1&$\Bigl(\dots\Bigr)$\\[.4cm]
%2& $\Bigl[\dots\Bigr]$\\[.4cm]
%3& $\Bigl\{\dots\Bigr\}$\\[.4cm]
%\bottomrule
%\end{tabular}
%}
%\caption{Precedenze}
%\label{Tab:precedenze}
%\end{table}
\section{Somme prodotti divisioni}
\label{sec:sommeprodottidivisioni}
\begin{table}[H]
\centering
\begin{tabular}{LL}
\toprule
a+b=b+a&b\cdot a=a\cdot b\\[.6cm]
a+a=2a&a\cdot a=a^2\\[.6cm]
a^n+a^m=a^n+a^m&a^n\cdot a^m=a^{n+m}\\[.6cm]
a+1=a+1&a\cdot1=a\\[.6cm]
1+a=1+a&1\cdot a=a\\[.6cm]
a+0=a&a\cdot 0=0\\[.6cm]
0+a=a&0\cdot a=0\\[.6cm]
\bottomrule
		\end{tabular}
	\caption{Somme, prodotti}
	\label{tab:prodottimonomi}
\end{table}
\begin{table}[H]
\centering
\begin{tabular}{LL}
\toprule
(a+b)(c+d)=ac+ad+bc+bd&(a-b)(a+b)=a^2-b^2\\[.6cm]
(a+b)^2=a^2+2ab+b^2&(a-b)^2=a^2-2ab+b^2\\[.6cm]
(a+b)^2=(a+b)(a+b)&(a-b)^2=(a-b)(a-b)\\[.6cm]
c(a+b)^2=c(a^2+2ab+b^2)& -(a-b+c)=-a+b-c\\[.6cm]
c(a-b)(a+b)=c(a^2-b^2)&a(b+c)=ab+bc\\[.6cm]
		(a+b)^3=a^3+3a^2b+3ab^2+b^3&(a-b)^3=a^3-3a^2b+3ab^2-b^3\\
\bottomrule
		\end{tabular}
	\caption{Prodotti notevoli}
	\label{tab:prodotti}
\end{table}
\begin{table}[H]
\centering
\begin{tabular}{LL}
\toprule
		a:b=\dfrac{a}{b} \quad b\neq 0 & \dfrac{1}{n}a=\dfrac{a}{n}\quad n\neq 0\\[.6cm]
\dfrac{a}{b}=a:b \quad b\neq 0 & \dfrac{a}{n}=\dfrac{1}{n}a\quad n\neq 0\\[.6cm]
\dfrac{a}{b}:\dfrac{c}{d}=\dfrac{a}{b}\cdot\dfrac{d}{c}\quad b\neq 0\quad c\neq 0\quad d\neq 0&\dfrac{a}{a}=1\quad a\neq 0\\[.6cm]
		
		\dfrac{a}{b}\cdot c=\dfrac{ac}{b} \quad b\neq 0&\dfrac{a}{b}\cdot\dfrac{c}{d}=\dfrac{a\cdot c}{b\cdot d} \quad b\neq 0\quad d\neq 0\\[.6cm]
		
		 -\dfrac{a+b}{c}=+\dfrac{-a-b}{c}\quad c\neq 0&\dfrac{a}{b-c}=-\dfrac{a}{c-b}\quad b\neq c  \\[.6cm]
		
		%\dfrac{a}{c}+\dfrac{b}{c}=\dfrac{a+b}{c}&\dfrac{a}{b}+\dfrac{c}{d}=\dfrac{[\dfrac{mcm(bd)}{b}\cdot a]+[\dfrac{mcm(bd)}{d}\cdot c]}{mcm(bd)}\\[.6cm]
\dfrac{a}{c}+\dfrac{b}{c}=\dfrac{a+b}{c}&\dfrac{a}{b}+\dfrac{c}{d}=\dfrac{[(mcm(bd)\div b)\cdot a]+[(mcm(bd)\div d)\cdot c]}{mcm(bd)}\\[.6cm]
 
\dfrac{a}{b}+c=\dfrac{a+bc}{b}&\dfrac{1}{a}(b+c)=\dfrac{b}{a}+\dfrac{c}{a}\\[.6cm]
		\bottomrule
		\end{tabular}
	\caption{frazioni}
	\label{tab:prodottifrazioni}
\end{table}
\begin{esempio}
Bisogna stare molto attenti alle divisioni 
\[\dfrac{3}{4}a^5b^6:\dfrac{3}{14}a^3b^3=\overbrace{\dfrac{3}{4}a^5b^6\cdot\dfrac{14}{3}a^3b^3}^{\text{Errore grave}}=\dfrac{7}{2}a^8b^9 \]
La procedura corretta è la seguente:
\[\dfrac{3}{4}a^5b^6:\dfrac{3}{14}a^3b^3=\overbrace{\dfrac{3a^5b^6}{4}\cdot\dfrac{14}{3a^3b^3    }}^{\text{Corretto}}=\dfrac{7}{2}a^2b^3 \]
\end{esempio}
