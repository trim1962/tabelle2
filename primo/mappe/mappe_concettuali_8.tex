\documentclass[preview=true]{standalone}
\input{../Mod_base/grafica}
\input{../Mod_base/base}
\input{impostazioni/impostazioniTikz}
\begin{document}
  \begin{tikzpicture}
    \node[main node2] (1) {Frazione};
    \node[main verb] (4) [right=2cm of 1]  {ha};
    \node[main verb] (3) [above=2cm of 4]  {ha};
     \node[main verb] (2) [above=2cm of 3]  {ha};
    \node[main verb] (5) [below= of 4]  {ha};
    \node[main verb] (6) [below= of 5]  {ha};
    \node[main node2] (7) [right= of 2] {Al denominatore c'è un numero formato non solo da potenze del 2 e del 5};
    \node[main node2] (8) [right= of 3] {Al denominatore c'è un numero che non è formato da potenze del 2 e del 5};
    \node[main node2] (9) [right= of 4] {Al denominatore c'è un numero che è formato da potenze del 2 e del 5};
    \node[main node2] (10) [right= of 5] {Denominatore decimale};
    \node[main node2] (11) [right= of 6] {Frazione impropria};
    \node[main verb] (12) [right= of 7]  {\'{e}};
   \node[main verb] (13) [right= of 8]  {\'{e}};
   %\node[main verb] (14) [right= of 9]  {\'{e}};
       \path let \p1=(9), \p2=(10) in node[main verb] (14) at (\x1+80,\y2/2+\y1/2){\'{e}};
   \node[main verb] (15) [right= of 11]  {\'{e}};
    \node[main node2] (16) [right= of 12] {Numero decimale periodico composto};
    \node[main node2] (17) [right= of 13] {Numero decimale periodico semplice};
    \node[main node2] (18) [right= of 14] {Numero decimale finito};
    \node[main node2] (19) [right= of 15] {Numero intero};
 
\foreach \x /\y in{1/2,1/3,1/4,1/5,1/6,2/7,3/8,4/9,5/10,6/11,7/12,8/13,9/14,10/14,11/15,12/16,13/17,14/18,15/19}
  \path[linea] (\x) edge node {} (\y);
%  
\end{tikzpicture}
\end{document}