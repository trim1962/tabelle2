\chapter{Esempi}
\section{Goniometria}
\label{sec:EsempiGoniometria}
\begin{table}[H]
	\caption{Trovare seno coseno tangente cotangente noti seno o coseno.}
	\label{tab:trovaresencosnoti}
\begin{enumerate}
	\item Prerequisiti 
\begin{itemize}
	\item I radicali
	\item Circonferenza goniometrica\index{Circonferenza!goniometrica}
	\item Seno\index{Funzione!Seno}, Coseno\index{Funzione!Coseno}, Tangente\index{Funzione!Tangente}, Cotangente\index{Funzione!Cotangente}
  \begin {align*}
	\cos\alpha=\pm\sqrt{1-\sin^2\alpha}\\
	\sin\alpha=\pm\sqrt{1-\cos^2\alpha}\\
	\tan\alpha=\dfrac{\sin\alpha}{\cos\alpha}\\
	\cot\alpha=\dfrac{1}{\tan\alpha}
	\end{align*}
\end{itemize}
  \item Scopo: Determinare le funzioni goniometriche dato il valore del seno o il coseno di un angolo.
  \item Testo: Dato $\sin\alpha=\dfrac{3}{5}$ con l'angolo $\alpha$ tale che $\ang{90}<\alpha<\ang{180}$ determinare: coseno, tangente e cotangente di $\alpha$
  \item Svolgimento: Si inizia con il determinare il coseno di un angolo successivamente la tangente e per finire la cotangente, quantità che possiamo conoscere noti seno e coseno. 
  \begin{enumerate}
	\item coseno: Dato che il coseno di un angolo è un numero relativo bisogna definire un segno ed un modulo
	\begin{enumerate}
	\item segno: Per valori dell'angolo  $\ang{90}<\alpha<\ang{180}$, secondo quadrante, il coseno è negativo. 
	\item modulo: $\cos\alpha=-\sqrt{1-\sin^2\alpha}=-\sqrt{1-\left(\dfrac{3}{5}\right)^2}=-\sqrt{1-\dfrac{9}{25}}=-\sqrt{\dfrac{25-9}{25}}=-\sqrt{\dfrac{16}{25}}=-\dfrac{4}{5}$
\end{enumerate}
	\item tangente:$\tan\alpha=\dfrac{\sin\alpha}{\cos\alpha}=\dfrac{\dfrac{3}{5}}{-\dfrac{4}{5}}=\dfrac{3}{5}\cdot\left(-\dfrac{5}{4} \right)=-\dfrac{3}{4}$
  \item cotangente:$\cot\alpha=\dfrac{1}{\tan\alpha}=\dfrac{1}{-\dfrac{3}{4}}=-\dfrac{4}{3}$
\end{enumerate}
 \end{enumerate}
\end{table}
%\clearpage
\begin{table}[H]
	\caption{Trovare seno coseno nota la tangente}
	\label{tab:sommadifangoli}
	\begin{enumerate}
		\item Prerequisiti 
		\begin{itemize}
			\item I radicali
			\item Circonferenza goniometrica
			\item Seno, Coseno, Tangente, Cotangente
			\begin {align*}
			\cos\alpha=\pm\dfrac{1}{\sqrt{1+\tan^2\alpha}}\\
			\sin\alpha=\pm\dfrac{\tan\alpha}{\sqrt{1+\tan^2\alpha}}\\
			\sin\alpha=\cos\alpha\cdot\tan\alpha\\
			\tan\alpha=\dfrac{1}{\cot\alpha}
		\end{align*}
	\end{itemize}
	\item Scopo: Determinare le funzioni goniometriche dato il valore della tangente di un angolo.
	\item Testo: Dato $\cot\alpha=\dfrac{2}{5}$ con l'angolo $\alpha$ tale che $\ang{180}<\alpha<\ang{270}$ determinare: coseno e seno di $\alpha$
	\item Svolgimento: Nell'esercizio non è nota la tangente dell'angolo quindi inizio a trovare la tangente dell'angolo, poi si passa al coseno infine al seno di un angolo.
	\begin{enumerate}
		\item tangente: $\tan\alpha=\dfrac{1}{\cot\alpha}=\dfrac{1}{\dfrac{2}{5}}=\dfrac{5}{2}$
		\item coseno: Dato che il coseno di un angolo è un numero relativo bisogna definire un segno ed un modulo.
		\begin{enumerate}
			\item segno: Per valori dell'angolo  $\ang{180}<\alpha<\ang{270}$, terzo quadrante, il coseno è negativo. 
			\item modulo:
			\begin{align*}
			\cos\alpha&=-\dfrac{1}{\sqrt{1+\tan^2\alpha}}
			=-\dfrac{1}{\sqrt{1+\left(\dfrac{5}{2}\right)^2}}=-\dfrac{1}{\sqrt{1+\dfrac{25}{4}}}
			=-\dfrac{1}{\sqrt{\dfrac{4+25}{4}}}=-\dfrac{1}{\sqrt{\dfrac{29}{4}}}\\
			&=-\dfrac{1}{\dfrac{\sqrt{29}}{\sqrt{4}}}=-\dfrac{1}{\dfrac{\sqrt{29}}{2}}=-\dfrac{2}{\sqrt{29}}=-\dfrac{2\sqrt{29}}{29}
			\end{align*}
		\end{enumerate}
		\item seno: per determinare il valore del seno ho due vie: utilizzare il valore del coseno appena determinato o calcolarlo direttamente 
		\begin{enumerate}
			\item primo caso: $\sin\alpha=\cos\alpha\cdot\tan\alpha=-\dfrac{2\sqrt{29}}{29}\cdot\dfrac{5}{2}=-\dfrac{5\sqrt{29}}{29}$
			\item secondo caso: Dato che il seno è un numero relativo bisogna determinarne segno e modulo
			\begin{enumerate}
				\item segno:  Per valori dell'angolo  $\ang{180}<\alpha<\ang{270}$, terzo quadrante, il seno è negativo.
				\item modulo:
				\begin{align*}
				\sin\alpha&=-\dfrac{\tan\alpha}{\sqrt{1+\tan^2\alpha}}
				=\dfrac{\dfrac{5}{2}}{\sqrt{1+\left(\dfrac{5}{2}\right)^2}}
				=-\dfrac{\dfrac{5}{2}}{\sqrt{1+\dfrac{25}{4}}}
				=-\dfrac{\dfrac{5}{2}}{\sqrt{\dfrac{4+25}{4}}}
				=-\dfrac{\dfrac{5}{2}}{\sqrt{\dfrac{29}{4}}}
				=-\dfrac{\dfrac{5}{2}}{\dfrac{\sqrt{29}}{\sqrt{4}}}\\
				&=-\dfrac{\dfrac{5}{2}}{\dfrac{\sqrt{29}}{2}}
				=-\dfrac{5}{2}\dfrac{2}{\sqrt{29}}
				=-\dfrac{5\sqrt{29}}{29}
				\end{align*} %$\sin\alpha=-\dfrac{\tan\alpha}{\sqrt{1+\tan^2\alpha}}=\dfrac{\dfrac{5}{2}}{\sqrt{1+\left(\dfrac{5}{2}\right\}}}=-\dfrac{\dfrac{5}{2}}{\sqrt{1+\dfrac{25}{4}}}=-\dfrac{\dfrac{5}{2}}{\sqrt{\dfrac{4+25}{4}}}=-\dfrac{\dfrac{5}{2}}{\sqrt{\dfrac{29}{4}}}=-\dfrac{\dfrac{5}{2}}{\dfrac{\sqrt{29}}{\sqrt{4}}}=-\dfrac{\dfrac{5}{2}}{\dfrac{\sqrt{29}}{2}}=-\dfrac{5}{2}\dfrac{2}{\sqrt{29}}=-\dfrac{5\sqrt{29}}{29}$
			\end{enumerate}
		\end{enumerate}
	\end{enumerate}
\end{enumerate}
\end{table}

 %FINE TERZO
