\chapter{Funzione sinusoidale}
\section{Definizioni}
\label{cha:FunzioneSinusoidale}
\begin{definizionet}{Funzione sinusoidale}{}
	Chiamo funzione sinusoidale una funzione del tipo:
	\[ y=R\sin(\omega t+\phi)\]
	\begin{itemize}
		\item $R$ è l'ampiezza positiva
		\item $\omega$ è la pulsazione positiva
		\item $\phi$ è lo sfasamento $-\pi\leq\phi\leq\pi$
		
	\end{itemize}
\end{definizionet}
\begin{osservazionet}{Periodicità}{}
	La funzione\[ y=R\sin(\omega t+\phi)\]è periodica di periodo\[T=\dfrac{2\pi}{\omega}\]
\end{osservazionet}
infatti\[ y=R\sin[\omega(t+T)+\phi]=R\sin[\omega(t+\dfrac{2\pi}{\omega})+\phi]=R\sin(\omega
t+2\pi+\phi)=R\sin(\omega
t+\phi)\]
Datato che il seno ha periodo \[2\pi\]
\begin{definizionet}{Frequenza}{}
	Se $T$ indica il periodo allora  $f=\dfrac{1}{T}$ è la frequenza.
\end{definizionet}
\section{Andamento funzione sinusoidale}
Considero la funzione sinusoidale\[y=R\sin\omega t\] Dato che la funzione è periodica, ne traccio il grafico nell'intervallo \[[0\leq t\leq T]\] come nella figura\nobs\vref{fig:FunzioneSinusoidale1}

La funzione assume il suo massimo valore $R$ per $t=\dfrac{T}{4}$ mentre il suo valore minimo è per $t=\dfrac{3}{4}T$. Questo si può velocemente verificare osservando che $\sin x=1$ quando $x=\dfrac{\pi}{2}$ quindi 
\begin{align*}
\omega t=&\dfrac{\pi}{2}\\
\intertext{ma}
\omega=&\dfrac{2\pi}{T}
\intertext{quindi}
t\dfrac{2\pi}{T}=&\dfrac{\pi}{2}\\
t=&\dfrac{T}{4}
\end{align*} 
In maniera analoga si dimostra per il punto di minimo. Dal grafico è evidente che la funzione è positiva per valori di $t$ compresi tra zero e metà periodo cioè: \[[0\leq t\leq \dfrac{T}{2}]\] Mentre per \[[\dfrac{T}{2}\leq t\leq T]\] assume valori negativi. 
\begin{figure}
	\centering
	\includestandalone[width=7.5cm]{terzo/FunzSinuo/sinuo1}
	\caption{Funzione sinusoidale}
	\label{fig:FunzioneSinusoidale1}
\end{figure}
\begin{definizionet}{Anticipo e ritardo di fase}{}
	Consideriamo le funzioni \[y=R\sin\omega t\] e\[y=R\sin(\omega t +\phi)\] Queste funzioni hanno la stessa ampiezza $R$ e 
	
\end{definizionet}
\begin{figure}
	\begin{subfigure}[b]{.5\linewidth}
		\centering
		\includestandalone[width=7.5cm]{terzo/FunzSinuo/sinuo2P}
		\caption{Anticipo di fase}\label{fig:FunzioneSinusoidaleAnticipo1}
	\end{subfigure}%
	\begin{subfigure}[b]{.5\linewidth}
		\centering
		\includestandalone[width=7.5cm]{terzo/FunzSinuo/sinuo2A}
		\caption{Ritardo di fase}\label{fig:FunzioneSinusoidalePosticipo1}
	\end{subfigure}
	\captionof{figure}{Fasi}
	\label{fig:sinuoAnticipoPosticipo}
\end{figure}