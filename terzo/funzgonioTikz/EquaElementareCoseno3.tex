\documentclass[preview=true]{standalone}
\input{../Mod_base/grafica}
\input{../Mod_base/base}
%\usetikzlibrary{...}
\begin{document}
		\begin{tikzpicture}[>=triangle 45]
		% draw the coordinates
		\pgfmathsetmacro{\raggio}{4};
		\pgfmathsetmacro{\pangolo}{180};
		\pgfmathsetmacro{\sangolo}{{-\pangolo}};
		\pgfmathsetmacro{\mraggio}{\raggio/3};
		\pgfmathsetmacro{\sraggio}{1.9*\raggio};
		% draw the unit circle
		\draw[->] (0,-\raggio-\mraggio) -- (0,\raggio+\mraggio) node[above,fill=white] {$y$};
		\draw[->] (-\raggio-\mraggio,0) -- (\raggio+\mraggio,0) node[right,fill=white] {$x$};
		\draw (-\raggio,-\raggio-\mraggio) -- (-\raggio,\raggio+\mraggio) ;
		\draw[thick] (0,0) circle(\raggio);
		\coordinate [label= below left:$O$] (OO)at(0,0);
		
		\coordinate (PX)  at ({\raggio*cos(\pangolo},0);
		
		\node at (PX) [label=above right:$m$]{};
		
		\fill [color=black] (OO) circle (1pt);
		
		\fill [color=black] (PX) circle (1pt);
		\draw [->] (0:\sraggio/4) arc (0:\sangolo:\sraggio/4) ;
		\draw (\sangolo/2:\sraggio/4) node[ below right]  {$-\ang{180}$};
		\draw [->] (0:\sraggio/4) arc (0:\pangolo:\sraggio/4) ;
		\draw (\pangolo/2:\sraggio/4) node[ above right]  {$+\ang{180}$};
		
		\end{tikzpicture}
	\end{document}