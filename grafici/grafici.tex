\begin{figure}
	\begin{subfigure}[b]{.5\linewidth}
		\centering
		\includestandalone[width=5cm]{terzo/funzgonioTikz/cosenodefinizione}
		\caption{Coseno definizione}\label{sub:graficicosenodef}
	\end{subfigure}%
	\begin{subfigure}[b]{.5\linewidth}
		\centering
		\includestandalone[width=7.5cm]{terzo/funzgonioTikz/cosenografico}
		\caption{Coseno grafico}\label{sub:graficicosenograf}
	\end{subfigure}
	\captionof{figure}{Coseno}
	\label{tab:graficifunzcos}
\end{figure}
\begin{figure}
	\begin{subfigure}[b]{.5\linewidth}
		%		\centering\includegraphics[scale=0.35]{senoalpha-crop}
		\centering
		\includestandalone[width=5cm]{terzo/funzgonioTikz/senodefinizione}
		\caption{Seno definizione}\label{sub:graficisenodef}
	\end{subfigure}%
	\begin{subfigure}[b]{.5\linewidth}
		\centering
		\includestandalone[width=7.5cm]{terzo/funzgonioTikz/senografico}
		\caption{Seno grafico}\label{sub:graficisenograf}
	\end{subfigure}
	\captionof{figure}{Seno}
	\label{tab:graficifunseno}
\end{figure}
\begin{figure}
	\centering
	\includestandalone[width=8.5cm]{terzo/funzgonioTikz/andamentoseno1}
	\captionof{figure}{Andamento seno $\ang{0}<\alpha<\ang{180}$}\label{fig:graficigraficiAndamentoSeno1}
\end{figure}%
\begin{figure}
	\centering
	\includestandalone[width=8.5cm]{terzo/funzgonioTikz/andamentoseno2}
	\captionof{figure}{Andamento seno $\ang{180}<\alpha<\ang{360}$}\label{fig:graficiAndamentoSeno2}
\end{figure}%
\begin{figure}
	\begin{subfigure}[b]{.5\linewidth}
		\centering\includestandalone[width=0.6\textwidth]{terzo/funzgonioTikz/segnocoseno}
		\caption{Segno coseno}\label{fig:graficiSegnoCoseno}
	\end{subfigure}%
	\begin{subfigure}[b]{.5\linewidth}
		\centering\includestandalone[width=0.6\textwidth]{terzo/funzgonioTikz/segnoseno}
		\caption{Segno seno}\label{fig:graficiSegnoSeno}
	\end{subfigure}
	\begin{subfigure}[b]{.5\linewidth}
		\centering\includestandalone[width=0.6\textwidth]{terzo/funzgonioTikz/segnotangente}
		\caption{Segno tangente}\label{fig:graficiSegnoTangente}
	\end{subfigure}%
	\begin{subfigure}[b]{.5\linewidth}
		\centering\includestandalone[width=0.6\textwidth]{terzo/funzgonioTikz/segnocotangente}
		\caption{Segno cotangente}\label{fig:graficiSegnoCotangente}
	\end{subfigure}
	\captionof{figure}{Segno funzioni goniometriche}
	\label{tab:graficisegnofunzionigoniometriche}
\end{figure}
\begin{figure}
	\begin{subfigure}[b]{.5\linewidth}
		\centering
		\includestandalone[width=5cm]{terzo/funzgonioTikz/tangentedefinizione}
		\caption{Tangente definizione}\label{fig:graficiTangenteDefinizione}
	\end{subfigure}%
	\begin{subfigure}[b]{.5\linewidth}
		\centering\includestandalone[width=7.5cm]{terzo/funzgonioTikz/tangentegrafico}
		\caption{Tangente grafico}\label{fig:graficiTangenteGrafico}
	\end{subfigure}
	\captionof{figure}{Tangente}
	\label{tab:graficifunztg}
\end{figure}
\begin{figure}
	\centering
	\includestandalone[width=8.5cm]{terzo/funzgonioTikz/tangenteandamento1}
	\captionof{figure}{Andamento tangente $\ang{0}<\alpha<\ang{180}$}\label{fig:graficiAndamentoTangente1}
\end{figure}%
\begin{figure}
	\centering
	\includestandalone[width=8.5cm]{terzo/funzgonioTikz/tangenteandamento2}
	\captionof{figure}{Andamento tangente $\ang{180}<\alpha<\ang{360}$}\label{fig:graficiAndamentoTangente2}
\end{figure}%

\begin{figure}
	\begin{subfigure}[b]{.5\linewidth}
		\centering
		\includestandalone[width=5cm]{terzo/funzgonioTikz/cotangentedefinizione}
		\caption{Cotangente}\label{fig:graficiCotangenteDefinizione}
	\end{subfigure}%
	\begin{subfigure}[b]{.5\linewidth}
		\centering\includestandalone[width=7.5cm]{terzo/funzgonioTikz/cotangentegrafico}
		\caption{Cotangente grafico}\label{fig:graficiCotangenteGrafico}
	\end{subfigure}
	\captionof{figure}{Tangente}
	\label{tab:graficifunztg}
\end{figure}
\begin{figure}
	\centering
	\includestandalone[width=8.5cm]{terzo/funzgonioTikz/cotangenteandamento1}
	\captionof{figure}{Andamento cotangente $\ang{0}<\alpha<\ang{180}$}\label{fig:graficiAndamentoCotangente1}
\end{figure}%
\begin{figure}
	\centering
	\includestandalone[width=8.5cm]{terzo/funzgonioTikz/cotangenteandamento2}
	\captionof{figure}{Andamento cotangente $\ang{180}<\alpha<\ang{360}$}\label{fig:graficiAndamentoCotangente2}
\end{figure}%
\begin{figure}
	\centering
	\begin{tikzpicture}[>=triangle  45]
	% draw the coordinates
	
	\pgfmathsetmacro{\raggio}{4};
	\pgfmathsetmacro{\mraggio}{\raggio/6};
	
	\pgfmathsetmacro{\sraggio}{1.9*\raggio};
	\draw[->] (0,-\raggio/2-\mraggio/2) -- (0,\raggio+\mraggio) node[above,fill=white] {$y$};
	\draw[->] (-\raggio/2-\mraggio/2,0) -- (\raggio+\mraggio,0) node[right,fill=white] {$x$};
	\node(OO)at(0,0) [label= above left:$O$] {};
	\foreach \x/ \y/\arco  in {
		55/5/{\alpha}%,
	}
	{
		\draw[->] (\sraggio/\y,0 ) arc (0:\x:\sraggio/\y);% node[above ] {$\arco$};
		\node (aa) at   ({cos(\x/2} , {sin(\x/2} ) [label=above:$\arco$] {};
		\draw(0,0) -- ({\raggio*cos(\x} , {\raggio*sin(\x})node[midway ]{a} node [above]{P};
		\draw[dashed] ({\raggio*cos(\x} , 0 ) -- ({\raggio*cos(\x} , {\raggio*sin(\x} )node[midway ]{c}  ;
		\draw [dashed](0, {\raggio*sin(\x} ) -- ({\raggio*cos(\x} , {\raggio*sin(\x} );
		\draw[dashed] (0 , 0 ) -- (0 , {\raggio*sin(\x} )node [above]{K};
		\draw [dashed](0, 0 ) -- ({\raggio*cos(\x} , 0 )node[midway ]{b} node [above]{H};;
	}
	\draw plot[domain=0:90,smooth] ({\raggio*cos(\x} , {\raggio*sin(\x});
	\end{tikzpicture}
	\caption{Relazione fondamentale goniometria}
	\label{fig:graficirelFondGonio}
\end{figure}
\begin{figure}
	\begin{subfigure}[b]{.5\linewidth}
		\centering\includestandalone[width=0.6\textwidth]{terzo/funzgonioTikz/CosenoNotoSeno1}
		\caption{}\label{fig:graficiCosenoNotoSeno1}
	\end{subfigure}%
	\begin{subfigure}[b]{.5\linewidth}
		\centering\includestandalone[width=0.6\textwidth]{terzo/funzgonioTikz/CosenoNotoSeno2}
		\caption{}\label{fig:graficiCosenoNotoSeno2}
	\end{subfigure}
	\begin{subfigure}[b]{.5\linewidth}
		\centering\includestandalone[width=0.6\textwidth]{terzo/funzgonioTikz/CosenoNotoSeno3}
		\caption{}\label{fig:graficiCosenoNotoSeno3}
	\end{subfigure}%
	\begin{subfigure}[b]{.5\linewidth}
		\centering\includestandalone[width=0.6\textwidth]{terzo/funzgonioTikz/CosenoNotoSeno4}
		\caption{}\label{fig:graficiCosenoNotoSeno4}
	\end{subfigure}
	\captionof{figure}{Coseno noto seno}
	\label{fig:graficiCosenoNotoSenoEs1}
\end{figure}
\begin{figure}
	\begin{subfigure}[b]{.5\linewidth}
		\centering\includestandalone[width=0.6\textwidth]{terzo/funzgonioTikz/senoNotoCoseno1}
		\caption{}\label{fig:graficisenoNotoCoseno1}
	\end{subfigure}%
	\begin{subfigure}[b]{.5\linewidth}
		\centering\includestandalone[width=0.6\textwidth]{terzo/funzgonioTikz/senoNotoCoseno2}
		\caption{}\label{fig:graficisenoNotoCoseno2}
	\end{subfigure}
	\begin{subfigure}[b]{.5\linewidth}
		\centering\includestandalone[width=0.6\textwidth]{terzo/funzgonioTikz/senoNotoCoseno3}
		\caption{}\label{fig:graficisenoNotoCoseno3}
	\end{subfigure}%
	\begin{subfigure}[b]{.5\linewidth}
		\centering\includestandalone[width=0.6\textwidth]{terzo/funzgonioTikz/senoNotoCoseno4}
		\caption{}\label{fig:graficisenoNotoCoseno4}
	\end{subfigure}
	\captionof{figure}{Seno Noto Coseno}
	\label{fig:graficisenoNotoCosenoEs1}
\end{figure}
\section{Angoli associati}
\label{sec:graficigoniometriaAngoliAssociati}
\subsection{Angoli supplementari}
\begin{figure}
	\centering
	\includestandalone[width=8.5cm]{terzo/funzgonioTikz/angoliassociati1}
	\caption{Angoli supplementari $\alpha$ $\ang{180}-\alpha$}
	\label{fig:graficiAngoliAssociatisupplementari}
\end{figure}
\begin{figure}
	\centering
	\includestandalone[width=8.5cm]{terzo/funzgonioTikz/angoliassociati2}
	\caption{Angoli che differiscono di $\ang{180}$ $\alpha$ e $\ang{180}+\alpha$}
	\label{fig:graficiAngoliAssociatidiff180}
\end{figure}
\subsection{Angoli esplementari}
\begin{figure} %[H]
	\centering
	\includestandalone[width=8.5cm]{terzo/funzgonioTikz/angoliassociati3}
	\caption{Angoli esplementari $\alpha$ e $\ang{360}-\alpha$}
	\label{fig:graficiAngolidif360}
\end{figure}
\subsection{Angoli opposti}
\begin{figure} %[H]
	\centering
	\includestandalone[width=8.5cm]{terzo/funzgonioTikz/angoliopposti}
	\caption{Angoli opposti}
	\label{fig:graficiangoliopposti}
\end{figure}
\subsection{Angoli complementari}
\begin{figure} %[H]
	\centering
	\includestandalone[width=8.5cm]{terzo/funzgonioTikz/angolicomplementari1}
	\caption{Angoli complementari $\alpha$ e  $\ang{90}-\alpha$}
	\label{fig:graficiangolicomplementari1}
\end{figure}
\subsection{Angoli la cui differenza è $\ang{90}$}
\begin{figure} %[H]
	\centering
	\includestandalone[width=8.5cm]{terzo/funzgonioTikz/angolicomplementari2}
	\caption{Angoli che differiscono di $\ang{90}$, $\alpha$ e $\ang{90}+\alpha$}
	\label{fig:graficiangolicomplementari2}
\end{figure}
\subsection{Angoli la cui somma è $\ang{270}$}
\begin{figure} %[H]
	\centering
	\includestandalone[width=8.5cm]{terzo/funzgonioTikz/angolicomplementari3}
	\caption{Angoli la cui somma è $\ang{270}$,  $\alpha$ e $\ang{270}-\alpha$}
	\label{tab:graficiangolicomplementari3}
\end{figure}
\subsection{Angoli la cui differenza è $\ang{270}$}
\begin{figure} %[H]
	\centering
	\includestandalone[width=8.5cm]{terzo/funzgonioTikz/angolicomplementari4}
	\caption{Angoli la cui differenza è $\ang{270}$, $\alpha$ e $\ang{270}+\alpha$}
	\label{tab:graficiangolicomplementari4}
\end{figure}
